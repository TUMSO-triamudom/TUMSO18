\documentclass[11pt,a4paper]{article}

\newcommand{\tumsoTime}{13:00 น. - 16:00 น.}
\newcommand{\tumsoRound}{2}

\usepackage{../style/tumso}

\begin{document}

\begin{problem}{ขโมยของข้ามมิติ}{standard input}{standard output}{1 second}{512 megabytes}{100}

จอมโจร 1112 ได้ขโมยสมบัติล้ำค่าแห่งจักรวาลนี้ไปแล้ว และกำลังจะนำไปเก็บไว้ที่ฐานลับของเขาในตำแหน่ง $(x_s,y_s)$ บนมิติแห่งหนึ่งซึ่งมิตินี้มีความกว้าง $M$ สูง $N, (1 \leq x_s \leq N,1 \leq y_s\leq M)$ และแต่ละช่อง $(x,y)$ ในมิติจะมีตัวเลข $d_{x,y}$ ซึ่งมีกฎดังนี้
\begin{enumerate}
\item 
ถ้า $d_{x,y}>0$ เราจะถือว่าหมายเลข $d_{x,y} = Mx' + y'-1$ โดยที่ $(x',y')$ คือช่องในมิติที่จอมโจรจะถูกย้ายตำแหน่งไปหากเดินทางมาถึง $(x,y)$

\item 	ถ้า $d_{x,y} \leq 0$ เราจะถือว่าถ้าจอมโจรเดินทางมาถึงช่อง $(x,y)$ แล้วจอมโจรจะสามารถเดินทางไปช่อง $(x',y')$ ใดๆก็ได้ที่ $|x-x' |+|y-y' |=|d_{x,y}|$

\end{enumerate}

เนื่องจากจอมโจรได้หลบหนีมาจากมิติที่ห่างไกลจึงสามารถเริ่มเดินทางไปฐานลับได้จากแค่ขอบของมิตินี้ ซึ่งก็คือช่อง $(1,y)$, $(x,1)$, 
$(N, y)$, $(x, M)$; $1 \leq x \leq N,1 \leq y \leq M$ แต่เขาก็กำลังแข่งขันกับเวลาว่าจะถูกตำรวจอวกาศตามตัวทันหรือไม่เขาจึงไหว้วานให้คุณซึ่งเก่งในด้านการเขียนโปรแกรมมากมาช่วยเขาในการหาระยะทางที่สั้นที่สุดที่จะพาเขาไปหาฐานลับของเขาโดยเริ่มต้นจากช่องใดก็ได้บนขอบของมิตินี้

\InputFile
บรรทัดแรก – จำนวนเต็ม $N,M,x_s,y_s (1 \leq N,M \leq 1000) $
อีก $N$ บรรทัดประกอบด้วยจำนวนเต็ม $M$ จำนวนแทน $d_{x,y}$ สำหรับแต่ละแถวของมิติ $(-N-M \leq d_{x,y} < MN)$

รับประกันว่าค่าของ $d_{x,y}$ จะเป็นค่าที่มีความหมายตามกฎ 1 และ 2

\OutputFile
จำนวนเต็มหนึ่งจำนวนแสดงระยะทางที่น้อยที่สุดเพื่อที่จะไปถึงฐานลับ ถ้าไม่สามารถเดินทางไปถึงฐานลับได้ให้ตอบ -1

\Scoring
ชุดทดสอบจะถูกแบ่งเป็น 2 ชุด จะได้คะแนนในแต่ละชุดก็ต่อเมื่อโปรแกรมให้ผลลัพธ์ถูกต้องในชุดทดสอบย่อยทั้งหมด

\begin{description}

\item[ชุดที่ 1 (37 คะแนน)] จะมี $ 1 \leq M, N \leq 400$

\item[ชุดที่ 2 (63 คะแนน)] ไม่มีเงื่อนไขเพิ่มเติม 

\end{description}

\Examples

\begin{example}
\exmp{5 5 3 3
5 5 5 5 5
5 17 5 5 5
21 5 0 11 5
5 18 5 5 5
5 5 5 5 5
}{5}%
\exmp{9 9 8 8
-3 43 49 40 59 -3  0 26 46
 0 58 19 88 20  9  0 -4 -3
-3  0 72 -4 23 68 10 87 53
82 81  0  0 77 27 47 75 -4
-4  0 -4 79 41 74 88  9 37
 0 61 65 -4 -3 -3  0 37  0
33 -4 87 -4 37 28 40 -3 88
72 -3 22 70 45 13 31  0 70
58 31 82 -3 47 75 -3 67 -4
}{4}%
\end{example}

\Note

\textbf{ตัวอย่างที่ 1}

*** ตัวอย่างเส้นทาง ***

*** 3:1 4:2 3:4 2:2 3:3 ***

\textbf{ตัวอย่างที่ 2}

*** ตัวอย่างเส้นทาง ***

*** 1:5 6:6 5:4 8:8 ***

\end{problem}

\end{document}
