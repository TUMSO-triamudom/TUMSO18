\documentclass[11pt,a4paper]{article}

\newcommand{\tumsoTime}{09:00 น. - 12:00 น.}
\newcommand{\tumsoRound}{1}

\usepackage{../style/tumso}

\begin{document}
\vspace*{\fill}%
\noindent
\begin{center}
{\Large \textbf{เฉลย}}

{\Large \textbf{การแข่งขันคณิตศาสตร์และวิทยาศาสตร์ระหว่างโรงเรียนครั้งที่ 18}}

{\Large \textbf{(18\textsuperscript{th} Triam Udom Mathematics and Science Olympiad)}} 

{\Large\textbf{วิชาคอมพิวเตอร์ รอบที่ 1}}

{\Large\textbf{วันที่ 9 มกราคม 2563 เวลา 09:00 น. - 12:00 น.}}

\begin{tabular}{ |c|c|c|c|c|c|  }
  \hline
  \textbf{ID โจทย์} & ชื่อโจทย์ & Time & Memory & คะแนนชุดทดสอบย่อย & รวม (คะแนน)\\
  \hline
  A-Tulympic & TU Lympics & 1 s & 256 MB & 30 70 & 100\\
  B-interdimensional-thief & ขโมยของข้ามมิติ& 1 s & 256 MB & 37 63 & 100\\
  C-zombieland & Zombie Land & 1 s & 256 MB & 15 35 50 & 100\\
  D-ezproblem & โจทย์สุดง่าย(มั้ง) & 1 s & 256 MB & 25 75 & 100\\
  E-sneakey-getaway & เนียนให้ผ่าน & 1 s & 256 MB & 30 70 & 100\\
  F-abgift & ab gift & 1 s & 256 MB & 20 80 & 100\\
  \hline
\end{tabular}

\end{center}
\vfill
\pagebreak


\begin{problem}{TU Lympics}{standard input}{standard output}{1 seconds}{256 megabytes}{100}

โจทย์ข้อนี้เป็นการทดสอบความสามารถในการ implement อัลกอริทึม เพราะฉะนั้น สามารถเขียนโค้ดตามที่โจทย์สั่งได้เลย
\begin{itemize}
\item รับข้อมูลคะแนนมาเก็บไว้ใส่ array 
\item สร้าง array ไว้สำหรับเก็บคะแนนของผู้เข้าแข่งขันแต่ละคน (เริ่มต้นที่ $0$ คะแนน)
\item สำหรับการแข่งขันแต่ละรอบ ให้สร้าง array ที่มีเลข $1$ ถึง $N$ ขึ้นมา แทนหมายเลขผู้เข้าแข่งขันในรอบนั้น แล้วเอามาเรียงลำดับจากน้อยไปมาก โดยให้เปรียบเทียบจากคะแนนของหมายเลขนั้นก่อน ถ้าคะแนนเท่ากันจึงเปรียบเทียบตัวเลขผู้เข้าแข่งขัน (หรืออาจจะเก็บเป็น pair ของคะแนนและหมายเลข แล้วใช้ฟังก์ชันเรียงลำดับข้อมูลก็ได้)
\item ดำเนินการเพิ่มคะแนนให้ผู้เข้าแข่งขันตามลำดับที่เรียงไว้
\item เมื่อทำครบทุกรอบแล้ว ให้เรียงลำดับคะแนนจากมากไปน้อย (วิธีคล้ายกับเรียงลำดับเวลาในขั้นตอนที่แล้ว) แล้วตอบว่าเพื่อนได้อันดับที่เท่าไร
\end{itemize}

หากใช้วิธีการเรียงข้อมูลที่ไม่มีประสิทธิภาพ เช่น Bubble Sort, Insertion Sort, Selection Sort จะได้เพียงชุดทดสอบกลุ่มที่ 1 เท่านั้น จะได้คะแนนเต็มก็ต่อเมื่อเขียนฟังก์ชันเรียงข้อมูลใน $O(n \log n)$ หรือใช้ฟังก์ชัน sort ที่มีในภาษา C++ อยู่แล้ว

\end{problem}

\pagebreak
%%%%%%%%%%%%%%%%%%%%%%%%%%%%%%%%%%%%%%%%%%%%%%%%%%%%%%%%%%%%%%%%%%%%%%%%%%%%%%%%%%%%%%%%%%%%

\begin{problem}{ขโมยของข้ามมิติ}{standard input}{standard output}{1 seconds}{256 megabytes}{100}

โจทย์ข้อนี้เป็นโจทย์ breadth first search หาเส้นทางสั้นสุดในกราฟตารางสองมิติพื้นฐาน โดยเงื่อนไขเส้นทางข้อที่ 2 (กรณี $d_{x,y} \leq 0$) จะต้องแจกแจงอย่างระมัดระวัง นั่นคือแจกแจงเฉพาะ node $(x',y')$ ที่มีระยะห่างจาก $(x,y)$ ไป $|d_{x,y}|$ พอดีเท่านั้น

อนึ่ง ถึงแม้ว่า Time Complexity จะเป็น $O(N^3)$ (เพราะกรณีที่ $d_{x,y} \leq 0$ อาจจะชี้ไปได้ถึง $O(N)$ node) ในทางปฏิบัติ จำนวนเส้นจากแต่ละ node ไม่ได้มีมากครบถึง $N=1250$

\end{problem}

\pagebreak
%%%%%%%%%%%%%%%%%%%%%%%%%%%%%%%%%%%%%%%%%%%%%%%%%%%%%%%%%%%%%%%%%%%%%%%%%%%%%%%%%%%%%%%%%%%%

\begin{problem}{Zombie Land}{standard input}{standard output}{1 seconds}{256 megabytes}{100}

ขั้นแรก จะต้องหาให้ได้ก่อนว่ามีตึกใดเป็นตึกอันตรายบ้าง ตามโจทย์ ตึกอันตรายคือตึกใด ๆ ที่อยู่ในเส้นทางที่สั้นที่สุดจากตึก $S$ ไปตึก $E$

ในกรณีที่มีเพียงเส้นทางเดียว การหาตึกอันตรายทำได้ไม่ยากนัก เพียงแค่ใช้ Dijkstra's Algorithm เริ่มต้นจากตึก $S$ ไปยังตึก $E$ พร้อมเก็บข้อมูล parent node ไว้ ทำให้สามารถ trace เส้นทางจากตึก $E$ ย้อนกลับมาหาตึก $S$ ได้

ในกรณีที่มีหลายเส้นทาง จะต้องใช้ข้อสังเกตเพิ่มเติมคือ ตึก $x$ จะเป็นตึกอันตรายได้ก็ต่อเมื่อ ระยะทางสั้นสุดจากตึก $S$ ไปตึก $x$ บวกกับระยะทางสั้นสุดจากตึก $E$ ไปตึก $x$ เท่ากับระยะทางสั้นสุดจากตึก $S$ ไปตึก $E$ นั่นคือ $dist(S,x)+dist(E,x) = dist(S,E)$ ดังนั้น ให้ใช้ Dijkstra's Algorithm หาเส้นทางสั้นสุดจากตึก $S$ ไปยังทุกตึก และจากจุด $E$ ไปยังทุกตึก แล้วใช้เงื่อนไขดังกล่าวเพื่อตรวจสอบว่าแต่ละตึกเป็นตึกอันตรายหรือไม่

เมื่อระบุตึกอันตรายทั้งหมดได้แล้ว ให้ใช้ Dijkstra's Algorithm โดยเริ่มต้นจากตึกอันตรายทุกตึกพร้อมกัน (เซต distance เป็น 0 แล้วนำใส่ priority queue ทุกตึกเลย) ไปยังตึกอื่น ๆ ทั้งหมด จะทำให้ได้ระยะทางสั้นสุดจากตึกอันตรายใด ๆ ไปยังตึกอื่น สามารถนำมาตอบคำถามทั้งหมด $Q$ คำถามได้ทันที

\end{problem}

\pagebreak
%%%%%%%%%%%%%%%%%%%%%%%%%%%%%%%%%%%%%%%%%%%%%%%%%%%%%%%%%%%%%%%%%%%%%%%%%%%%%%%%%%%%%%%%%%%

\begin{problem}{โจทย์สุดง่าย(มั้ง)}{standard input}{standard output}{1 seconds}{256 megabytes}{100}

สำหรับโจทย์ประเภทนี้ ควรเริ่มโดยการลองแจกแจงกรณี $n=1,2,3,\dots$ ก่อนเพื่อดูว่ามีแบบรูปใด ๆ หน้าสัมประสิทธิ์หรือไม่ เนื่องจากเราสนใจเพียงแค่สัมประสิทธิ์ของ $x^2$ เท่านั้น เราสามารถคูณพหุนามได้โดยข้ามพจน์ $x^3, x^4, \dots$ ได้เลย
\begin{align*}
f(0,a) &= x-a \\
f(1,a) &= x^2 + (-2a)x + a \\
f(2,a) &= \dots + (2a+4a^2)x^2 + (-4a^2)x + a \\
f(3,a) &= \dots + (4a^2+8a^3+16a^4)x^2 + (-8a^3)x + a \\
f(4,a) &= \dots + (8a^3+16a^4+32a^5+64a^6)x^2 + (-16a^4)x + a \\
\end{align*}

สังเกตว่า $f(n,a)$ จะมีสัมประสิทธิ์หน้า $x^2$ ทั้งหมด $n$ พจน์บวกกัน โดยเริ่มจาก $(2a)^{n-1}$ ไปถึง $(2a)^{2n-2}$ นั่นคือ $\sum_{i=n-1}^{2n-2} (2a)^i$

จากสูตรอนุกรมเรขาคณิต $1+r^1+r^2+\dots+r^{n-1} = \frac{r^n-1}{r-1}$
\begin{align*}
\sum_{i=n-1}^{2n-2} (2a)^i &= \sum_{i=0}^{2n-2} (2a)^i - \sum_{i=0}^{n-2} (2a)^i \\
&= \frac{(2a)^{2n-1}}{2a-1} - \frac{(2a)^{n-1}}{2a-1} \\
&= \left((2a)^{2n-1} - (2a)^{n-1}\right)\left(2a-1\right)^{-1}
\end{align*}

การคิดเลขยกกำลังเมื่อ $n$ มีค่ามากถึง $10^{18}$ สามารถทำได้ด้วยเทคนิค Binary Exponentiation

เนื่องจากโจทย์ต้องการให้เราตอบใน $10^9+7$ เราจะต้องใช้ Fermat's Little Theorem ช่วย นั่นคือ $(2a-1)^{-1} \equiv (2a-1)^{p-2} \mod p$ เราสามารถหาค่าของ $(2a-1)^{p-2}$ โดยใช้ Binary Exponentiation เช่นเดียวกันกับก่อนหน้านี้

\end{problem}

\pagebreak
%%%%%%%%%%%%%%%%%%%%%%%%%%%%%%%%%%%%%%%%%%%%%%%%%%%%%%%%%%%%%%%%%%%%%%%%%%%%%%%%%%%%%%%%%%%

\begin{problem}{เนียนให้ผ่าน}{standard input}{standard output}{1 seconds}{256 megabytes}{100}

เพื่อความสะดวก เราจะแปลงข้อมูลใหม่โดยนำข้อมูลทุกตัวลบด้วย $50\%$ ของความจุผู้โดยสาร ดังนั้นเราจะได้ $X_i$ เท่ากับจำนวนผู้โดยสารที่เกินมาจาก $50\%$ (ถ้าติดลบก็คือขาด) ดังนั้น โจทย์ข้อนี้จึงกลายเป็น จงหาวิธีแบ่งข้อมูลออกเป็นส่วนที่ติดกันโดยแต่ละส่วนจะต้องมีผลรวมเกิน $0$ ให้ได้จำนวนส่วนมากที่สุด โจทย์การแบ่งส่วนข้อมูล (Partitioning Problem) เป็นโจทย์พื้นฐานใน dynamic programming สามารถแก้ได้ดังนี้

นิยาม $dp_i$ เป็นจำนวนส่วนที่มากที่สุดที่เป็นไปได้ เมื่อพิจารณาปัญหาย่อยที่มีข้อมูลเพียง $i$ ตัวแรกเท่านั้น
เราจะต้องเลือกว่าส่วนสุดท้ายควรเริ่มที่ตำแหน่งใด แล้วจึงแบ่งส่วนข้อมูลที่เหลือต่อแบบ recursive
หากเลือกให้ข้อมูลช่วงสุดท้ายเริ่มต้นที่ตำแหน่ง $j+1$ (นั่นคือจะเหลือข้อมูลเพียง $j$ ตัวเท่านั้นโดย $j < i$) เราจะได้คำตอบเป็น $dp_j+1$ โดยมีเงื่อนไขคือ ผลรวมของ $X$ ตั้งแต่ตำแหน่งที่ $j+1$ ถึง $i$ จะต้องมากกว่า $0$

เราจะต้องเลือกว่าช่วงสุดท้ายควรเริ่มต้นที่ตำแหน่งใดดี หากเลือกตำแหน่ง $j$ ($1 \leq j < i$) ช่วงสุดท้าย (ช่วง $j$ ถึง $i$) จะต้องมีผลรวมเกิน $0$ ถ้าเลือกช่วงนี้เป็นช่วงสุดท้าย จำนวนช่วงทั้งหมดจะเท่ากับ $dp_{j-1}+1$ (ทั้งนี้ กรณีที่ช่วงสุดท้ายครอบคลุม array ทั้งหมด นั่นคือ $j=1$ ให้พิจารณา $dp_0 = 0$ เพราะจะได้คำตอบเป็น $1$ ช่วงพอดี)

เพื่อความสะดวก กำหนดให้ $\sigma(i) = \sum_{j=1}^{i} X_j$ (prefix sum) เราสามารถเขียน recurrence formula ได้ดังนี้
$$dp_i = \max_{0 \leq j < i, \sigma(i)-\sigma(j) > 0} \left( dp_j + 1 \right)$$
(กำหนดให้ $dp_0 = X_0 = 0$, กรณีที่ไม่มีค่าใดที่ตรงเงื่อนไข $\max$ กำหนดให้ $dp_i = -\infty$)

จากสูตรดังกล่าว สามารถคำนวณคำตอบได้ภายในเวลา $O(n^2)$ ทำให้ได้คะแนนในชุดทดสอบที่ 1 ไป

สำหรับชุดทดสอบที่ 2 ให้สังเกตเงื่อนไขว่า $\sigma(i)-\sigma(j) > 0$ คือ $\sigma(i)>\sigma(j)$ ดังนั้น หากเรามอง $\sigma(i)$ เป็น array แล้ว recurrence formula นี้ก็คือการแก้โจทย์คลาสสิก Longest Increasing Subsequence นั่นเอง วิธี greedy และ binary search จะทำให้สามารถแก้โจทย์ข้อนี้ได้ใน $O(n \log n)$ และได้คะแนนเต็มในที่สุด

\end{problem}

\pagebreak
%%%%%%%%%%%%%%%%%%%%%%%%%%%%%%%%%%%%%%%%%%%%%%%%%%%%%%%%%%%%%%%%%%%%%%%%%%%%%%%%%%%%%%%%%%%

\begin{problem}{ab gift}{standard input}{standard output}{1 second}{256 megabytes}{100}

สมมุติว่าเลือกช่วงที่ $l$ ถึง $r$ เราจะสามารถจัดรูปคำตอบได้ดังนี้
$$\sum_{i=l}^{r} \left( a_i \cdot \frac{\prod_{j=l}^{r} b_j}{b_i} \right) = \left(\sum_{i=l}^{r} \frac{a_i}{b_i}\right)\left(\prod_{i=l}^{r} b_i\right)$$

หากเก็บ prefix sum โดย $\sigma(i) = \sum_{j=1}^{i} \frac{a_j}{b_j}$ และ prefix product โดย $\pi(i) = \prod_{j=1}^{i} b_j$ (ควรใช้ตัวแปรประเภท double ในการเก็บข้อมูล อย่าลืมว่าโจทย์ให้ $10000b_i$ มา ไม่ใช่ $b_i$) เราสามารถหาคำตอบได้โดยการทดลองทุกช่วง $l \leq r$ ที่เป็นไปได้ คำนวณคำตอบ $\left( \sigma(r)-\sigma(l-1) \right) \left( \frac{\pi(r)}{\pi(l-1)} \right)$ หาว่าช่วงใดได้คำตอบมากที่สุด ได้ Time Complexity เป็น $O(N^2)$ ทำให้ได้คะแนนในชุดทดสอบกลุ่มแรก

ชุดทดสอบกลุ่มที่สองจะต้องใช้การสังเกต สมมุติว่าตอนนี้เรามีช่วง $l$ ถึง $r$ อยู่แล้ว เราจะพิจารณาเพิ่มข้อมูลตัวที่ $r+1$ เข้ามาก็ต่อเมื่อมันทำให้คำตอบดีขึ้น เพื่อความสะดวกจะเขียนให้ $s = \sigma(r)-\sigma(l-1)$ (ผลรวมของ $\frac{a_i}{b_i}$ ในช่วงตอนนี้) และ $p = \frac{\pi(r)}{\pi(l-1)}$ (ผลคูณของ $b_i$ ในช่วงตอนนี้)
\begin{align*}
sp &< \bigg(s+\frac{a_{r+1}}{b_{r+1}}\bigg)(p \cdot b_{r+1}) \\
s &< \bigg(s+\frac{a_{r+1}}{b_{r+1}}\bigg)(b_{r+1}) \\
s &< s b_{r+1} + a_{r+1} \\
s(1-b_{r+1}) &< a_{r+1}
\end{align*}

จากข้อมูล $a_i + 10000b_i = 10000 \implies a_i = 10000(1-b_i)$ เราสามารถเขียนได้ดังนี้
\begin{align*}
s(1-b_{r+1}) &< 10000(1-b_{r+1}) \\
s &< 10000
\end{align*}

นั่นคือ เราควรขยายช่วงทางด้านขวาไปเรื่อย ๆ ตราบใดที่ผลรวมของ $\frac{a_i}{b_i}$ ในช่วงนั้นน้อยกว่า $10000$ อยู่ ดังนั้น การเขียนโค้ดข้อนี้สามารถทำได้ด้วยวิธี two-pointer โดยเริ่มต้น ให้ $l=r=1$ แล้วลองขยับ $r$ เพิ่มขึ้นเรื่อย ๆ ให้ได้มากที่สุด เมื่อเสร็จแล้วจึงเพิ่มเป็น $l=2$ แล้วขยับ $r$ อีก, เพิ่มเป็น $l=3$ แล้วขยับ $r$ อีก, $\dots$ เช่นนี้ไปเรื่อย ๆ โดยระหว่างที่ทำให้จดคำตอบที่มากที่สุดไว้

วิธีนี้จะใช้ time complexity เพียง $O(N)$ เท่านั้น ทำให้ได้คะแนนเต็มในข้อนี้

\end{problem}

\end{document}

