\documentclass[11pt,a4paper]{article}

\newcommand{\tumsoTime}{13:00 น. - 16:00 น.}
\newcommand{\tumsoRound}{2}

\usepackage{../style/tumso}

\lstdefinestyle{customc}{
  belowcaptionskip=1\baselineskip,
  breaklines=true,
  xleftmargin=\parindent,
  language=C,
  showstringspaces=false,
  basicstyle=\footnotesize\ttfamily,
  keywordstyle=\bfseries\color{green!40!black},
  commentstyle=\itshape\color{purple!40!black},
  identifierstyle=\color{blue},
  stringstyle=\color{orange},
}

\lstdefinestyle{customasm}{
  belowcaptionskip=1\baselineskip,
  frame=L,
  xleftmargin=\parindent,
  language=[x86masm]Assembler,
  basicstyle=\footnotesize\ttfamily,
  commentstyle=\itshape\color{purple!40!black},
}

\lstset{escapechar=@,style=customc}

\begin{document}
\vspace*{\fill}%
\noindent
\begin{center}
{\Large \textbf{เฉลย}}

{\Large \textbf{การแข่งขันคณิตศาสตร์และวิทยาศาสตร์ระหว่างโรงเรียนครั้งที่ 18}}

{\Large \textbf{(18\textsuperscript{th} Triam Udom Mathematics and Science Olympiad)}} 

{\Large\textbf{วิชาคอมพิวเตอร์ รอบที่ 2}}

{\Large\textbf{วันที่ 9 มกราคม 2563 เวลา 13:00 น. - 16:00 น.}}

\begin{tabular}{ |c|c|c|c|c|c|  }
  \hline
  \textbf{ID โจทย์} & ชื่อโจทย์ & Time & Memory & คะแนนชุดทดสอบย่อย & รวม (คะแนน)\\
  \hline
  G-final-crisis & Final Crisis & 1 s & 256 MB & 30 70 & 100\\
  H-forest-resorts & Forest Resorts & 1 s & 256 MB & 25 25 50 & 100\\
  I-mathmath & math math & 1 s & 256 MB & 15 35 50 & 100\\
  J-isekai & Isekai No Hajine & 1 s & 256 MB & 10 20 70 & 100\\
  K-precious-treasure & สมบัติล้ำค่า & 1 s & 256 MB & 10 25 65 & 100\\
  L-autocomplete & Autocomplete & 1 s & 256 MB & 100 & 100\\
  \hline
\end{tabular}

\end{center}
\vfill
\pagebreak

\begin{problem}{Final crisis}{standard input}{standard output}{1 second}{256 megabytes}{100}

ในข้อนี้สังเกตว่าการอ่านหนังสือของเทพเอิร์ธจะมีผลทำให้จำนวนหน้าของวิชาที่อ่านมากกว่านั้นน้อยที่สุดเท่าที่จะเป็นไปได้ ดังนั้นแต่ละคำถามในโจทย์ข้อนี้จึงกลายเป็นการ minimize maximum ของ prefix sum สองชุด โดยที่แต่ละชุดต้องมีสมาชิกรวมกันเท่ากับ $k$

ในแต่ละคำถาม วิธีตรงไปตรงมาในการหาคำตอบคือการไล่หาจากจำนวนหนังสือที่อ่านในวิชาหนึ่ง เช่น หาก $k = 3$ เราจะไล่ชีววิทยา 0 เล่ม ประวัติศาสตร์ 3 เล่ม, ชีววิทยา 1 เล่ม ประวัติศาสตร์ 2 เล่ม, ชีววิทยา 2 เล่ม ประวัติศาสตร์ 1 เล่ม และ ชีววิทยา 3 เล่ม ประติศาสตร์ 0 เล่ม แล้วคำนวณค่า minimum ในทุก ๆ เคส 

วิธีนี้ใช้เวลา $O(qn)$ และจะได้ 20 คะแนน

ในการปรับปรุง เราจะใช้เทคนิคซึ่งคล้าย ๆ กับ binary search ก่อนอื่นเราจะทำการ "เดา" จำนวนหนังสือในวิชา ๆ หนึ่งมาก่อน เรียกจำนวนนี้ว่า $x$ (ซึ่งทำให้เรารู้จำนวนหนังสือในอีกวิชาหนึ่ง: $k-x$) จากนั้นเราจะมาเทียบจำนวนหน้าของสองวิชาจากจำนวนเล่มที่เราเดา (สามารถทำได้ใน $O(1)$ โดย Quicksum) สมมติว่าวิชาแรกมีจำนวนหน้าเยอะกว่า เราก็จะรู้ทันทีว่าถ้าเราเลือกอ่านหนังสือวิชาแรกมากกว่า $x$ เล่ม คำตอบจะไม่มีทางดีขึ้น ทำให้เราสามารถทิ้งทุกค่าที่มากกว่า $x$ ไปได้เลย

วิธีนี้ใช้เวลา $O(q \log n)$ และจะได้ 100 คะแนน
\end{problem}

\pagebreak
%%%%%%%%%%%%%%%%%%%%%%%%%%%%%%%%%%%%%%%%%%%%%%%%%%%%%%%%%%%%%%%%%%%%%%%%%%%%%%%%%%%%%%%%%%%%

\begin{problem}{Forest Resorts}{standard input}{standard output}{1 second}{256 megabytes}{100}

สังเกตก่อนว่ารีสอร์ตนั้นควรจะถูกสร้างบนโหนดที่เป็น leaf (โหนดที่มีดีกรี 1) ของต้นไม้เท่านั้น เพราะหากสร้างที่อื่น เราสามารถย้ายลงมาสร้างที่ leaf เพื่อที่จะทำให้จำนวนถนนที่สร้างขึ้นนั้นมากกว่าเดิมได้

ในตอนเริ่มแรกนั้นให้เราจินตนาการว่าเราได้สร้างรีสอร์ตไว้บน leaf ทุก ๆ leaf แล้ว (ทำให้เราจำเป็นที่จะต้องสร้างถนนทุกเส้น) จากนั้นในการหาคำตอบสำหรับจำนวนรีสอร์ตที่น้อยลงมา เราจะเลือกแบบ greedy คือเลือกทำลายรีสอร์ตที่เมื่อทำลายแล้วจะทำให้ต้องลบถนนออกน้อยที่สุดจาก configuration ก่อนหน้า

ในการทำเช่นนี้ เราสามารถใช้ priority queue ในการตัดสินว่าจะเลือกทำลายรีสอร์ตที่โหนดไหน รวมกับการ update จำนวนถนนที่จะหายไปหากทำลายรีสอร์ตที่โหนดนั้น ๆ 

เมื่อจบแล้ว $ans[x]$ จะเก็บว่าหากจะสร้างรีสอร์ต $x$ ที่ จะต้องลบ edge ออกจาก tree ทั้งหมดกี่เส้น

อัลกอริธึมนี้ใช้เวลา $O(n \log n)$ และจะได้ 100 คะแนน

\end{problem}

\pagebreak
%%%%%%%%%%%%%%%%%%%%%%%%%%%%%%%%%%%%%%%%%%%%%%%%%%%%%%%%%%%%%%%%%%%%%%%%%%%%%%%%%%%%%%%%%%%%

\begin{problem}{Math Math}{standard input}{standard output}{1 second}{256 megabytes}{100}

คำตอบของข้อนี้สำหรับแต่ละค่า $k$ คือ $567k + 9$ วิธีแก้โจทย์ข้อนี้มีหลายวิธี อาทิเช่น ผู้เข้าแข่งขันอาจจะ implement BigNum ในการหา pattern ของคำตอบ แต่เราได้ทำการเขียนบทพิสูจน์ของสูตรนี้ให้ผู้อ่านได้เข้าใจ

กำหนดให้  $p$  เป็นจำนวนเฉพาะคี่และ $a$ เป็นจำนวนเต็มที่ $(a,p) =1$ สัญลักษณ์เลอฌ็องดร์ {\em(Legendre symbol)} $\left(\frac{a}{p}\right)$ นิยามดังนี้


\[ \left(\frac{a}{p}\right) =
  \begin{cases}
    1 & \quad \text{ถ้า } a \text{ เป็นส่วนตกค้างกำลังสองของ } p\\
    -1  & \quad \text{ถ้า } a \text{ ไม่เป็นส่วนตกค้างกำลังสองของ } p
  \end{cases}
\]


จาก $\left( \dfrac{10}{127} \right) = -1$ จะได้ว่า $\dfrac{10^{63}+1}{127}$ เป็นจำนวนเต็ม

ให้ $\dfrac{1}{127} = \dfrac{a_1}{10} + \dfrac{a_2}{100} + ... + \dfrac{a_n}{10^n} + ...$ เมื่อ $a_i$ เป็นเลขโดด

จะได้ว่า $\dfrac{10^{63}}{127} = k + \dfrac{a_{64}}{10} + \dfrac{a_{65}}{100} + ... + \dfrac{a_{n+63}}{10^n}+ ...$  เมื่อ $k$ เป็นจำนวนเต็มบวก

ดังนั้น $\dfrac{10^{63}+1}{127} - k = \dfrac{a_1+a_{64}}{10} + \dfrac{a_2+a_{65}}{100} +...+ \dfrac{a_n + a_{n+63}}{10^n} + ... $

ให้ RHS = $m$ จากการ bound จะได้ว่า $0< m< 2$ ดังนั้น $m = 1$
จะได้ $a_i + a_{i+63} = 9$ สำหรับ $i = 1, 2, ...$

จาก $m^{\phi(n)} \equiv 1\ (\textrm{mod}\ n)$  เมื่อ $(m, n) = 1$ และ $n$ เป็นจำนวนเฉพาะ และ $\phi(127) = 126$ จะได้ว่า $10^{126} \equiv 1\ (\textrm{mod}\ 127)$ ซึ่งหมายความว่า $ 127\ \mid \ 10^{126k}-1

ให้ $\dfrac{10^{126k}-1}{127} = \overline{a_1a_2a_3...a_n}$

เราจะได้ว่า $\dfrac{10^{126k+3}-1000}{127} = \overline{a_1a_2a_3...a_n000}$

และ $\dfrac{10^{126k+3}+143}{127} = \overline{a_1a_2a_3...a_n009}$

จาก $\dfrac{1}{127} = \dfrac{a_1}{10} + \dfrac{a_2}{100} + ... + \dfrac{a_n}{10^n} + ...$ เมื่อ $a_i$ เป็นเลขโดด

จาก $\dfrac{10^{126k}}{127} = \overline{a_1a_2...a_{126k}} + \dfrac{a_{126k+1}}{10} + ...$

$\dfrac{10^{126k}-1}{127} = \overline{a_1a_2...a_{126k}} + \sum_{i = 1}^{\infty}\dfrac{a_{i+126k}\cdot a_i}{10^i}$

เนื่องจาก $\sum_{i = 1}^{\infty}\dfrac{a_{i+126k}\cdot a_i}{10^i}$ เป็นจำนวนเต็ม และ $-1< \sum_{i = 1}^{\infty}\dfrac{a_{i+126k}\cdot a_i}{10^i} < 1$

ดังนั้น $\sum_{i = 1}^{\infty}\dfrac{a_{i+126k}\cdot a_i}{10^i} = 0$

ดังนั้นผลรวมเลขโดด $\dfrac{10^{126k}-1}{127} = a_1 + a_2 + ...+ a_{126k}$

จาก $a_i + a_{i+63} = 9$ จะได้ว่า $a_1 + a_2 + ...+ a_{126k} = 9\cdot 63\cdot k$

ดังนั้นผลรวมเลขโดด $\dfrac{10^{126k+3}+143}{127} = 9\cdot 63 \cdot k + 9 = 567k + 9$ 
\end{problem}

\pagebreak
%%%%%%%%%%%%%%%%%%%%%%%%%%%%%%%%%%%%%%%%%%%%%%%%%%%%%%%%%%%%%%%%%%%%%%%%%%%%%%%%%%%%%%%%%%%

\begin{problem}{Isekai No Hajime}{standard input}{standard output}{1 second}{256 megabytes}{100}

วิธีการแก้โจทย์ข้อนี้ที่ง่ายที่สุดคือการ simulate การพังทลายของหุบเขาและความสูงที่เปลี่ยนไป เพื่อหา damage ที่มากที่สุดที่เราสามารถทำได้ต่อราชาปีศาจในแต่ละวัน

ข้อสังเกตสำคัญ 3 ข้อในการแก้ปัญหาข้อนี้ คือ
\begin{enumerate}
\item สมมติปีศาจยิง laser ระดับที่ $x$ แล้ว ถ้าหากในคืนหลังจากนั้นปีศาจยิง laser ระดับที่ $y$ โดยที่ $y \geq x$ จะไม่เกิดอะไรขึ้น
\item  หากเราไม่พิจารณา laser ที่ "useless" จากข้อสังเกตข้อแรกแล้ว ช่วงของหุบเขาใด ๆ ที่โดนทำลายมาก่อนหน้านี้ จะโดนทำลายต่อไปในอนาคตเสมอ
\item  เราสามารถเปรียบการที่พื้นที่ช่วง ๆ หนึ่งถูกทำลาย เหมือนการ "เชื่อม" พื้นที่ข้าง ๆ ทั้งสองด้านเข้าด้วยกัน (ตราบใดที่ช่วง ๆ นั้นถูกทำลายแล้ว) 
\end{enumerate}

ข้อสังเกตทั้ง 3 ข้อนี้สามารถนำไปสู่อัลกอริธึมที่ใช้แก้โจทย์ปัญหาข้อนี้ได้ โดยในอัลกอริธึมของเรา เราจะทำการสร้าง event point ของการที่พื้นที่หนึ่งถูกทำลาย ยุบ และถูกเชื่อมรวมกัน เป็นการ merge สองพื้นที่เข้าด้วยกัน โดยใช้ disjoint set ในการเก็บข้อมูลนั้น ๆ และสำหรับแต่ละคืนเราจะคำนวณผลรวม damage ที่เราสามารถทำได้ต่อราชาปีศาจ เมื่อทำเช่นนี้แล้ว การตอบปัญหาแต่ละข้อก็เป็นการ binary search หาวัน ๆ แรกที่เราสามารถฆ่าราชาปีศาจได้สำเร็จนั่นเอง

วิธีนี้ใช้เวลา $O((Q+N) \log N)$ และจะได้ 100 คะแนน

\end{problem}

\pagebreak
%%%%%%%%%%%%%%%%%%%%%%%%%%%%%%%%%%%%%%%%%%%%%%%%%%%%%%%%%%%%%%%%%%%%%%%%%%%%%%%%%%%%%%%%%%%

\begin{problem}{สมบัติล้ำค่า}{standard input}{standard output}{1 second}{256 megabytes}{100}

ให้ $f(n)$ เป็นจำนวนวิธีการปูกระเบื้องเป็นทางยาวขนาด $2*n$ และให้ $g(n)$ เป็นจำนวนวิธีการปูกระเบื้องเป็นทางยาวขนาด $2*n$ โดยที่ทางยาวนี้ต้องนำไปต่อกระเบื้องสีดำ 1 ชิ้นที่ปลายข้างหนึ่ง และต้องเป็นไปตามกฎทุกอย่าง

เราจะให้ $f(0) = g(0) = 1$ (การที่ไม่ปูอะไรเลย ถือเป็นวิธีการปูกระเบื้องวิธีหนึ่ง) จากนี้เราจะความสัมพันธ์เวียนเกิดให้ $f(n)$ และ $g(n)$

เรามาพิจารณา $f(n)$ ก่อน ให้จินตนาการทางยาวขนาด $2*n$ ที่วางตามแนวนอนอยู่ การวางกระเบื้อง 2 ชิ้นทางด้านขวาจะเป็นไปได้ 3 กรณี

\begin{enumerate}
\item ปูกระเบื้องขาวทั้ง 2 ชิ้น 
\item ปูกระเบื้องขาว 1 ชิ้น ดำ 2 ชิ้น
\item ปูกระเบื้องดำทั้ง 2 ชิ้น (ซึ่งเป็นไปไม่ได้ เพราะจะผิดกฎ)
\end{enumerate}

หากเราปูกระเบื้องในวิธีแรกนั้น วิธีในการปูกระเบื้องที่เหลือ $n-1$ หลักจะเท่ากับ $f(n-1)$ เพราะการปูกระเบื้องสีขาวสองชิ้นจะไม่ทำให้การปูกระเบื้องที่เหลือมีสิทธิ์ผิดกฎเลย สำหรับวิธีที่สองวิธีการปูกระเบื้องจะเท่ากับ $g(n-1)$  ตามนิยามที่เราได้ให้ไปก่อนหน้านี้ และเนื่องจากเราสามารถปูกระเบื้องตามวิธีที่ 2 ได้ 2 วิธี จะได้ว่า

$$f(n) = f(n-1)+2g(n-1)$$

สำหรับ $g(n)$ นั้น เราต้องปูกระเบื้องโดยคิดว่ามีกระเบื้องสีดำวางไว้อยู่แล้ว สำหรับกระเบื้อง 2 ชิ้นที่ปลายนั้น 1 ชิ้นที่ติดกับกระเบื้องสีดำจะต้องเป็นสีขาว (มิเช่นนั้นจะผิดกฎ) จากนั้นจะเหลือ 2 กรณี

\begin{enumerate}
\item วางกระเบื้องสีดำ
\item วางกระเบื้องสีขาว
\end{enumerate}

ในกรณีแรกนั้นวิธีการปู $n-1$ หลักที่เหลือจะเท่ากับ $g(n-1)$ และในกรณีที่สองวิธีการปูจะเท่ากับ $f(n-1)$ ดังนั้น

$$g(n) = f(n-1)+g(n-1)$$

และคำตอบของเราจะเท่ากับ $f(n)$ หากเราคำนวณ $f(n)$ ไว้ก่อนตอบคำถาม วิธีนี้จะใช้เวลา $f(N + t)$ เมื่อ $N$ คือ $n$ ที่มากที่สุดในแต่ละคำถาม วิธีนี้จะได้ 35 คะแนน

เพื่อที่จะทำให้เร็วขึ้น เราสามารถเขียนสมการทั้งสองสมการข้างต้นได้ในรูปแบบต่อไปนี้
\[
\begin{bmatrix}
f(n)\\
g(n)
\end{bmatrix}
=
\begin{bmatrix}
1 & 2\\
1 & 1
\end{bmatrix}
\begin{bmatrix}
f(n-1)\\
g(n-1)
\end{bmatrix}
\]
จะได้ว่า 
\[
\begin{bmatrix}
f(n)\\
g(n)
\end{bmatrix}
=
\begin{bmatrix}
1 & 2\\
1 & 1
\end{bmatrix}^n
\begin{bmatrix}
1\\
1
\end{bmatrix}
\]
จากนี้จะเห็นได้ชัดว่า ปัญหาของเราคือการยกกำลัง matrix ให้เร็วกว่า $O(n)$ ซึ่งจริง ๆ แล้วสามารถทำได้ โดยใช้เทคนิค exponentiation by squaring ซึ่งทำการยกกำลัง matrix ได้ใน $O(\log n)$

วิธีนี้จะใช้เวลา $O(\log N + t)$ และจะได้ 100 คะแนน
\end{problem}

\pagebreak
%%%%%%%%%%%%%%%%%%%%%%%%%%%%%%%%%%%%%%%%%%%%%%%%%%%%%%%%%%%%%%%%%%%%%%%%%%%%%%%%%%%%%%%%%%%

\begin{problem}{Autocomplete}{standard input}{standard output}{1 second}{256 megabytes}{100}

การแก้ปัญหานี้จะสามารถแยกออกได้เป็นสองส่วนคือ
\begin{enumerate}
\item Scanning: เปลี่ยนจาก text/code เป็นลำดับของ token
\item Parsing: เปลี่ยนลำดับของ token เป็น parse tree
\end{enumerate}

เทคนิคที่สามารถใช้ได้ในการ scan และการ parse มีดังต่อไปนี้

\begin{enumerate}
\item token ใด ๆ คือหน่วยที่เกิดจากการต่อตัวอักษรบางส่วนเข้าด้วยกัน
\item เราไม่จำเป็นที่จะต้องสนใจเครื่องหมาย + หรือ - และเราสามารถยุบตัวเลขให้กลายเป็นเลขเดียวกันได้เลย เช่น 0
\item เราสามารถแยกชนิดของ token ออกได้ง่ายเลย คือ 0-9 เป็นตัวเลข a-z เป็น identifier และ A-Z เป็น keyword
\end{enumerate}

จากนั้นให้เราพิจารณาตาม สิ่งที่สามารถแทน # ได้โดยใช้ parse tree และรายชื่อของ identifier และ keyword ทั้งหมด
\end{problem}

\end{document}

