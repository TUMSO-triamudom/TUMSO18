\documentclass[11pt,a4paper]{article}

\newcommand{\tumsoTime}{09:00 น. - 12:00 น.}
\newcommand{\tumsoRound}{1}

\usepackage{../style/tumso}

\begin{document}

\begin{problem}{เนียนให้ผ่าน}{standard input}{standard output}{1 seconds}{256 megabytes}{100}


ไม่นะ อยู่ในช่วงที่สถานีรถไฟทูยาร์ปต้องได้รับการตรวจปริมาณผู้โดยสารขั้นต่ำประจำปี โดยทางสถานีรถไฟได้เก็บบันทึกปริมาณผู้โดยสารที่จะเอาไปให้สำนักงานใหญ่ดูหมดแล้ว แต่ปรากฏว่าบันทึกบางส่วนนั้นมีจำนวนผู้โดยสารไม่ผ่านเกณฑ์ที่จะต้องมากกว่า 50\% ของความจุผู้โดยสารสูงสุด ถ้าสำนักงานใหญ่เห็นสถานีจะต้องโดนปิดอย่างแน่นอน ดังนั้นทางสถานีรถไฟจึงคิดแผนการชั่วร้ายออกมาคือการรวมผลบันทึกปริมาณผู้โดยสารนั่นเอง!

	เนื่องจากจำนวนผู้โดยสารในบางบันทึกนั้นเกิน 50\% ไปมากพอที่จะเป็นตัวช่วยของบันทึกอื่นๆที่มีจำนวนผู้โดยสารไม่เพียงพอได้ แต่มีข้อแม้อยู่เล็กน้อย คือการรวมบันทึกนั้นจะต้องรวมบันทึกลำดับที่อยู่ติดกันเท่านั้น เพราะว่าวันที่ของไฟล์ที่รวมแล้วจะได้ไม่โดดไปมาซึ่งจะทำให้สำนักงานใหญ่สงสัย และต้องพยายามให้จำนวนครั้งที่รวมน้อยที่สุดด้วย เพราะจะได้มีบันทึกหลายๆบันทึกไปส่งให้สำนักงานใหญ่ดูได้

	จงหาจำนวนบันทึกที่มากที่สุดหลังจากรวมแล้ว ทางสถานีรถไฟทูยาร์ปจะได้ทราบว่าตนเองมีโอกาสรอดเนียนจากสำนักงานใหญ่เท่าไหร่ 

\InputFile
บรรทัดแรก ระบุจำนวนเต็ม $ n,m$ $( 1\leq n\leq10^5 ; 1\leq m\leq10^6)$ โดยที่ $n$ แทนจำนวนบันทึกและ $m$ แสดงความจุผู้โดยสารสูงสุดโดย $m$ จะเป็นเลขคู่เสมอ

บรรทัดที่สอง ระบุจำนวนเต็ม $n$ ตัว ระบุ $X_1, X_2, ..., X_n$ ($1\leq X_i\leq m
)$ โดยที่  $X_i$ แสดงจำนวนผู้โดยสารในบันทึกลำดับที่ $i$ รับประกันว่าหากรวมทุกบันทึกเข้าด้วยกันแล้วจะมีจำนวนผู้โดยสารเกิน 50\% 

\OutputFile
มีทั้งหมด $1$ บรรทัดระบุจำนวนบันทึกที่มากที่สุดหลังจากรวมแล้วโดยที่ทุกบันทึกจะต้องมีจำนวนผู้โดยสารมากกว่า 50\%

\Scoring
ชุดทดสอบจะถูกแบ่งเป็น 2 ชุด จะได้คะแนนในแต่ละชุดก็ต่อเมื่อโปรแกรมให้ผลลัพธ์ถูกต้องในชุดทดสอบย่อยทั้งหมด

\begin{description}

\item[ชุดที่ 1 (30 คะแนน)] จะมี $ 1\leq n\leq10^3$

\item[ชุดที่ 2 (70 คะแนน)] ไม่มีเงื่อนไขเพิ่มเติม

\end{description}

\Examples

\begin{example}
\exmp{5 100
60 48 20 90 49
}{2}%
\exmp{12 100
60 48 40 56 59 57 45 48 51 52 53 54
}{6}%
\end{example}

\Note
\begin{note}
\textbf{ตัวอย่างที่ 1}

รวมบันทึกที่ $1,2$ ได้ $108/200$ คน

รวมบันทึกที่ $3,4,5$ ได้ $159/300$ คน

\textbf{ตัวอย่างที่ 2}

รวมบันทึกที่ $1,2$ ได้ $108/200$ คน

รวมบันทึกที่ $3,4,5$ ได้ $115/300$ คน

รวมบันทึกที่ $6,7$ ได้ $102/200$ คน

รวมบันทึกที่ $8,9,10$ ได้ $151/300$ คน

บันทึกที่ $11,12$ ไม่ต้องรวม

\end{note}


\end{problem}

\end{document}