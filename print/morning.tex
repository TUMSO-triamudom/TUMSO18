\documentclass[11pt,a4paper]{article}
\documentclass{memoir}

\newcommand{\tumsoTime}{09:00 น. - 12:00 น.}
\newcommand{\tumsoRound}{1}

\usepackage{./tumso}

\begin{document}
\vspace*{\fill}%
\noindent
\begin{center}
{\Large \textbf{การแข่งขันคณิตศาสตร์และวิทยาศาสตร์ระหว่างโรงเรียนครั้งที่ 18}}

{\Large \textbf{(18\textsuperscript{th} Triam Udom Mathematics and Science Olympiad)}} 

{\Large\textbf{วิชาคอมพิวเตอร์ รอบที่ 1}}

{\Large\textbf{วันที่ 9 มกราคม 2563 เวลา 09:00 น. - 12:00 น.}}

\begin{tabular}{ |c|c|c|c|c|c|  }
  \hline
  \textbf{ID โจทย์} & ชื่อโจทย์ & Time & Memory & คะแนนชุดทดสอบย่อย & รวม (คะแนน)\\
  \hline
  A-Tulympic & TU Lympics & 1 s & 256 MB & 30 70 & 100\\
  B-interdimensional-thief & ขโมยของข้ามมิติ & 4 s & 512 MB & 37 63 & 100\\
  C-zombieland & Zombie Land & 1 s & 256 MB & 15 35 50 & 100\\
  D-ezproblem & โจทย์สุดง่าย(มั้ง) & 1 s & 256 MB & 25 75 & 100\\
  E-sneakey-getaway & เนียนให้ผ่าน & 1 s & 256 MB & 30 70 & 100\\
  F-abgift & ab gift & 1 s & 256 MB & 20 80 & 100\\
  \hline
\end{tabular}

\end{center}
\vfill
\pagebreak

{\Large \textbf{คำชี้แจงเกี่ยวกับระบบการแข่งขัน}}

\begin{enumerate}
  \item ผู้เข้าแข่งขันจะต้องล็อกอินเข้าสู่ระบบการแข่งขันด้วย Username และ Password ที่จัดเตรียมไว้ให้ภายในระบบ
  \item ผู้เข้าแข่งขันจะต้องเขียนโปรแกรมภาษา C, C++ ที่มีคุณลักษณะตามที่ระบุไว้ในโจทย์ แล้วอัพโหลด source code เพื่อให้เซิฟเวอร์ทำการประมวลผล
  \item ระบบจะแสดงผลคะแนนทันทีที่ประมวลผลเสร็จ (อาจมีความล่าช้าหากมีการส่งคำตอบเข้ามาในระบบเป็นจำนวนมาก)
  \item ผู้เข้าแข่งขันสามารถส่งคำตอบสำหรับโจทย์ 1 ข้อกี่ครั้งก็ได้คะแนนจะคิดจากผลรวมของคะแนนของชุดทดสอบย่อยทั้งหมดที่ทำผ่าน
  \item โปรแกรมจะต้องให้ทำงานภายในเวลาและหน่วยความจำที่กำหนด และให้ผลลัพธ์ถูกต้องจึงจะได้รับคะแนนในโจทย์ข้อนั้น 
  \item โจทย์แต่ละข้อจะถูกแบ่งเป็นชุดทดสอบย่อยที่มีขอบเขตข้อมูลนำเข้าแตกต่างกัน ถึงแม้โปรแกรมของผู้เข้าแข่งขันจะไม่สามารถทำงานได้ในทุกกรณี ผู้เข้าแข่งขันจะได้รับคะแนนของแต่ละชุดทดสอบย่อยที่สามารถทำได้ตามที่ระบุไว้ในโจทย์
  \item หากมีข้อสงสัยเกี่ยวกับโจทย์ หรือเกิดความขัดข้องกับระบบหรือคอมพิวเตอร์ที่ใช้ในการทำโจทย์ ให้ยกมือสอบถามผู้คุมสอบเท่านั้น
\end{enumerate}

{\Large \textbf{ข้อมูลเพิ่มเติมเกี่ยวกับการทำโจทย์}}

\begin{enumerate}
  \item โปรแกรมที่ส่งมาในระบบจะต้องรับข้อมูลนำเข้าผ่านทาง standard input และแสดงผลข้อมูลผ่านทาง standard output
  \item ภาษาที่เลือกใช้อาจส่งผลต่อความเร็วในการทำงานของโปรแกรม ทำให้ไม่สามารถใช้บางภาษาใน
  การแก้โจทย์บางข้อ (รับประกันว่าสามารถใช้ภาษา C++ ในการแก้โจทย์ได้ทุกข้อ)
  \item \textbf{โจทย์บางข้ออาจมีข้อมูลนำเข้าหรือข้อมูลส่งออกเป็นจำนวนมาก ควรเลือกใช้ฟังก์ชัน I/O ที่
  สามารถทำงานได้อย่างรวดเร็ว (เช่น ใช้ scanf/printf แทน cin/cout ในภาษา C++)}
\end{enumerate}

\pagebreak

\begin{problem}{TU Lympics}{standard input}{standard output}{1 seconds}{256 megabytes}{100}

ในที่สุดก็ถึงเวลาสำหรับงานกีฬาสี ทุกๆ คนต่างมีภาระหน้าที่ที่ต้องรับผิดชอบ ไม่ว่าจะเป็นทั้งพาเหรด สแตนด์ และอื่นๆ ทั้งเบื้องหน้าและเบื้องหลัง นอกจากนี้ โรงเรียนก็ได้คิดกฎและกติกาการแข่งขันแบบใหม่ ในหลายๆ ชนิดกีฬา รวมไปถึงการแข่งขันวิ่งด้วย โดยรายละเอียดของกฎและกติกาแบบใหม่นั้น มีดังนี้

ในการแข่งขันนั้นจะมีผู้เข้าแข่งขันทั้งหมด $N$ คน แต่ละคนจะมีหมายเลขประจำตัวตั้งแต่ $1$ ถึง $N$ โดยจะแข่งขันกันทั้งหมด $M$ รอบ ผู้ชนะคือผู้ที่ได้รับคะแนนสูงที่สุด หากมีผู้ที่ได้คะแนนสูงสุดมากกว่า $1$ คน ผู้ชนะคือผู้ที่มีหมายเลขประจำตัวน้อยที่สุด โดยวิธีการคิดคะแนน จะคิดเป็นรอบต่อรอบ กล่าวคือ ในแต่ละรอบ เราจะนำเวลาที่ผู้เข้าแข่งแต่ละคนใช้ในรอบนั้นๆ มาเรียงลำดับจากน้อยไปมาก ผู้เข้าแข่งขันคนใดที่ใช้เวลาน้อยที่สุด ก็จะได้คะแนนมากที่สุด คนถัดมาก็จะได้คะแนนลดลง เรื่อยไปจนถึงอันดับสุดท้าย ในกรณีที่มีผู้เข้าแข่งขันมากกว่าหนึ่งคนใช้เวลาเท่ากัน ผู้เข้าแข่งขันคนใดที่มีหมายเลขประจำตัวที่น้อยกว่า จะได้อันดับที่ดีกว่า

ในการแข่งขันครั้งนี้นั้น เพื่อนของคุณเป็นหนึ่งในผู้เข้าแข่งขัน และเขานั้นก็อยากจะรู้ว่าในการแข่งขันครั้งนี้นั้น เขาได้อันดับที่เท่าไหร่ แต่ด้วยเหตุการณ์ทางเทคนิคบางประการ จึงทำให้ผลการแข่งขันนั้นเกิดความล่าช้า เพื่อนของคุณทราบว่าคุณนั้นมีความสามารถในการเขียนโปรแกรมระดับยอดเยี่ยม จึงได้มาขอให้คุณไปช่วยฝ่ายสถิติ เขียนโปรแกรมเพื่อจัดอันดับในการแข่งขันวิ่งครั้งนี้ เพื่อที่เขาจะได้รู้ว่าเขานั้นได้อันดับที่เท่าไหร่ และในบางครั้ง เขานั้นก็อยากจะรู้ด้วยว่า เขานั้นได้คะแนนเท่าไหร่ 

\InputFile
ข้อมูลนำเข้านั้นมีด้วยกัน $N+2$ บรรทัด 

บรรทัดที่ $1$ ประกอบด้วยจำนวนเต็มบวก $N$ $M$ $X$ และ $T$ แทนจำนวนผู้เข้าแข่งขัน จำนวนรอบในการแข่งขัน หมายเลขประจำตัวผู้เข้าแข่งขันของเพื่อนของคุณ และประเภทของคำถามตามลำดับ
	
บรรทัดที่ $2$ ประกอบด้วยจำนวนเต็มบวกทั้งหมด $N$ ตัว แทนคะแนนที่ผู้เข้าแข่งขันจะได้รับในแต่ละรอบ เมื่อได้ลำดับที่ $1$ ไปจนถึงลำดับที่ $N$ ตามลำดับ โดยรับประกันว่าคะแนนจะเรียงลำดับจากมากไปน้อยเสมอ และจะไม่ซ้ำกัน

บรรทัดที่ $2+i$ ประกอบด้วยจำนวนเต็มบวกทั้งหมด $M$ ตัวแทนเวลาที่ผู้เข้าแข่งขันคนที่ $i$ ใช้ในรอบที่ $1$ จนถึงรอบที่ $M$ ในหน่วยวินาที  $(1 \leq i \leq N )$ 

และรับประกันว่าเวลาที่ผู้เข้าแข่งขันใช้ในแต่ละรอบจะไม่เกิน $10^9$ วินาที คะแนนที่ผู้เข้าแข่งขันจะได้ในแต่ละรอบจะมีค่าไม่เกิน $2 \cdot 10^8$ คะแนน

\OutputFile
มี $1$ บรรทัด 

หาก $T = 1$ ให้แสดงจำนวนเต็มบวก $1$ ตัว คืออันดับที่เพื่อนของคุณได้ 

หาก $T = 2$ ให้แสดงจำนวนเต็มบวก $2$ ตัว คืออันดับที่เพื่อนของคุณได้ และคะแนนที่เขาได้รับตามลำดับ โดยเว้นช่องว่าง $1$ ช่องระหว่างจำนวนเต็มบวกทั้งสอง

\Scoring
ชุดทดสอบจะถูกแบ่งเป็น $2$ ชุด จะได้คะแนนในแต่ละชุดก็ต่อเมื่อโปรแกรมให้ผลลัพธ์ถูกต้องในชุดทดสอบย่อยทั้งหมด

\begin{description}

\item[ชุดที่ 1 (30 คะแนน)]  $N \leq 200$ , $M \leq 5$ , $1 \leq X \leq N$ , $T=1$

\item[ชุดที่ 2 (70 คะแนน)]  $N\leq20000$  , $M\leq5$ , $1 \leq X \leq N $ , $T=1$ หรือ $2$

\end{description}

\Examples

\begin{example}
\exmp{5 3 4 1
10 7 5 2 1
12 24 18
17 20 19
30 12 13
10 15 22
20 22 21
}{2
}%
\exmp{7 4 2 2
19 18 16 12 8 5 2
20 11 21 32
21 20 20 17
20 14 19 20
17 30 19 22
25 18 15 26
40 10 20 30
25 22 22 18
}{2 51
}%
\end{example}

\end{problem}

\pagebreak
%%%%%%%%%%%%%%%%%%%%%%%%%%%%%%%%%%%%%%%%%%%%%%%%%%%%%%%%%%%%%%%%%%%%%%%%%%%%%%%%%%%%%%%%%%%%

\begin{problem}{ขโมยของข้ามมิติ}{standard input}{standard output}{4 seconds}{512 megabytes}{100}

จอมโจร 1112 ได้ขโมยสมบัติล้ำค่าแห่งจักรวาลนี้ไปแล้ว และกำลังจะนำไปเก็บไว้ที่ฐานลับของเขาในตำแหน่ง $(x_s,y_s)$ บนมิติแห่งหนึ่งซึ่งมิตินี้มีความกว้าง $M$ สูง $N, (1 \leq x_s \leq N,1 \leq y_s\leq M)$ และแต่ละช่อง $(x,y)$ ในมิติจะมีตัวเลข $d_{x,y}$ ซึ่งมีกฎดังนี้
\begin{enumerate}
\item 
ถ้า $d_{x,y}>0$ เราจะถือว่าหมายเลข $d_{x,y} = Mx' + y'-1$ โดยที่ $(x',y')$ คือช่องในมิติที่จอมโจรจะถูกย้ายตำแหน่งไปหากเดินทางมาถึง $(x,y)$

\item 	ถ้า $d_{x,y} \leq 0$ เราจะถือว่าถ้าจอมโจรเดินทางมาถึงช่อง $(x,y)$ แล้วจอมโจรจะสามารถเดินทางไปช่อง $(x',y')$ ใดๆก็ได้ที่ $|x-x' |+|y-y' |=|d_{x,y}|$

\end{enumerate}

เนื่องจากจอมโจรได้หลบหนีมาจากมิติที่ห่างไกลจึงสามารถเริ่มเดินทางไปฐานลับได้จากแค่ขอบของมิตินี้ ซึ่งก็คือช่อง $(1,y)$, $(x,1)$, 
$(N, y)$, $(x, M)$; $1 \leq x \leq N,1 \leq y \leq M$ แต่เขาก็กำลังแข่งขันกับเวลาว่าจะถูกตำรวจอวกาศตามตัวทันหรือไม่เขาจึงไหว้วานให้คุณซึ่งเก่งในด้านการเขียนโปรแกรมมากมาช่วยเขาในการหาระยะทางที่สั้นที่สุดที่จะพาเขาไปหาฐานลับของเขาโดยเริ่มต้นจากช่องใดก็ได้บนขอบของมิตินี้

\InputFile
บรรทัดแรก – จำนวนเต็ม $N,M,x_s,y_s (1 \leq N,M \leq 1250) $
อีก $N$ บรรทัดประกอบด้วยจำนวนเต็ม $M$ จำนวนแทน $d_{x,y}$ สำหรับแต่ละแถวของมิติ $(-N-M \leq d_{x,y} < MN)$

รับประกันว่าค่าของ $d_{x,y}$ จะเป็นค่าที่มีความหมายตามกฎ 1 และ 2

\OutputFile
จำนวนเต็มหนึ่งจำนวนแสดงระยะทางที่น้อยที่สุดเพื่อที่จะไปถึงฐานลับ ถ้าไม่สามารถเดินทางไปถึงฐานลับได้ให้ตอบ -1

\Scoring
ชุดทดสอบจะถูกแบ่งเป็น 2 ชุด จะได้คะแนนในแต่ละชุดก็ต่อเมื่อโปรแกรมให้ผลลัพธ์ถูกต้องในชุดทดสอบย่อยทั้งหมด

\begin{description}

\item[ชุดที่ 1 (37 คะแนน)] จะมี $ 1 \leq M, N \leq 400$

\item[ชุดที่ 2 (63 คะแนน)] ไม่มีเงื่อนไขเพิ่มเติม 

\end{description}

\Examples

\begin{example}
\exmp{5 5 3 3
5 5 5 5 5
5 17 5 5 5
21 5 0 11 5
5 18 5 5 5
5 5 5 5 5
}{5}%
\exmp{9 9 8 8
-3 43 49 40 59 -3  0 26 46
 0 58 19 88 20  9  0 -4 -3
-3  0 72 -4 23 68 10 87 53
82 81  0  0 77 27 47 75 -4
-4  0 -4 79 41 74 88  9 37
 0 61 65 -4 -3 -3  0 37  0
33 -4 87 -4 37 28 40 -3 88
72 -3 22 70 45 13 31  0 70
58 31 82 -3 47 75 -3 67 -4
}{4}%
\end{example}

\Note

\textbf{ตัวอย่างที่ 1}

*** ตัวอย่างเส้นทาง ***

*** 3:1 4:2 3:4 2:2 3:3 ***

\textbf{ตัวอย่างที่ 2}

*** ตัวอย่างเส้นทาง ***

*** 1:5 6:6 5:4 8:8 ***

\end{problem}

\pagebreak
%%%%%%%%%%%%%%%%%%%%%%%%%%%%%%%%%%%%%%%%%%%%%%%%%%%%%%%%%%%%%%%%%%%%%%%%%%%%%%%%%%%%%%%%%%%%

\begin{document}

\begin{problem}{Zombie Land}{standard input}{standard output}{1 seconds}{256 megabytes}{100}

มีเมืองอยู่เมืองหนึ่ง มีตึกทั้งสิ้น $N$ ตึก แต่ละตึกจะมีถนนเชื่อมอยู่ทั้งหมด $M$ สาย ไปยังอีกตึกหนึ่ง ซึ่งสามารถเดินทางไปกลับได้ โดยถนนเหล่านี้มีระยะเวลาที่ใช้ในการเดินอยู่ ถนนเหล่านี้จะเชื่อมตึกเข้าด้วยกัน โดยที่สำหรับคู่ตึกใดๆ จะสามารถเดินทางถึงกันผ่านระบบถนนเหล่านี้ได้เสมอ

มีการทดลองบางอย่างเกิดขึ้นที่เมือง $S$ ทำให้มีซอมบี้ระบาดที่เมืองนั้น หน่วยกู้ภัยจึงอพยพคนไปยังตึก $E$ ซึ่งในภายหลัง ซอมบี้รู้ว่าคนไปอยู่ที่ตึก $E$ กันหมด จึงพยายามเดินทางไปยังตึกตึกนั้น แต่ซอมบี้เองก็ไม่ได้โง่ รู้จักการเดินแบบที่จะใช้ระยะเวลาให้น้อยที่สุดด้วย และสำหรับตึกที่ซอมบี้เดินผ่าน ก็จะแพร่เชื้อใส่คนที่ยังอาศัยอยู่ในตึกนั้นด้วย

คุณเป็นหน่วยกู้ภัย เมื่อรู้ว่าซอมบี้รู้ที่อยู่ของคน เลยอยากอพยพคนหนี จึงอยากทราบว่าสำหรับตึก $v$ นั้น เดินจากตึกที่มีโอกาสมีซอมบี้มายังตึกนี้ จะใช้เวลาน้อยสุดเท่าไร เนื่องจากเส้นทางจาก $S$ ไป $E$ ที่สั้นที่สุดอาจมีหลายทาง คุณจึงอยากเตรียมตัวในกรณีที่แย่มี่สุดไว้ก่อน

\InputFile
ข้อมูลนำเข้ามีทั้งหมด $1 + M + 1 + Q$ บรรทัด

บรรทัดแรกประกอบด้วยจำนวนเต็ม $N$ $M$ $S$ และ $E$ $(1 \leq N, M \leq 2 \cdot 10^5, 1 \leq S, E \leq N)$ 

บรรทัดถัดมาอีก $M$ บรรทัดประกอบด้วย $u$ $v$ $w$ $(1 \leq u \neq v \leq N, 1 \leq w \leq 10^9)$ แทนถนนที่เชื่อมจากตึก $u$ ไปยังตึก $v$ โดยใช้ระยะเวลาในการเดินเท่ากับ $w$

บรรทัดถัดมาประกอบด้วย $Q$ $(1 \leq Q \leq 2 \cdot 10^5)$ แทนจำนวนตึกที่คุณต้องการตรวจสอบ

บรรทัดถัดมาอีก $Q$ บรรทัดประกอบด้วย $u$ $(1 \leq u \leq N)$ แทนหมายเลขตึกที่ต้องการตรวจสอบ โดยโปรแกรมจะต้องแสดงค่าออกมาตามที่โจทย์ได้กล่าวไว้

\OutputFile
มีทั้งหมด $Q$ บรรทัด แต่ละบรรทัดประกอบด้วยคำตอบของแต่ละคำถาม

\Scoring
ชุดทดสอบจะถูกแบ่งเป็น 3 ชุด จะได้คะแนนในแต่ละชุดก็ต่อเมื่อโปรแกรมให้ผลลัพธ์ถูกต้องในชุดทดสอบย่อยทั้งหมด

\begin{description}

\item[ชุดที่ 1 (15 คะแนน)] จะมี $ 1 \leq N \leq 10^3 $

\item[ชุดที่ 2 (35 คะแนน)] สำหรับคู่เมืองใดๆ จะมีเส้นทางที่ไปหากันได้เพียง $1$ เส้นทางเท่านั้น

\item[ชุดที 3 (50 คะแนน)] ไม่มีเงื่อนไขเพิ่มเติม

\end{description}

\Examples

\begin{example}
\exmp{8 8 1 8
1 2 7
2 3 6
2 5 2
3 7 5
5 7 9
7 8 3
3 4 1
5 6 4
2
4
6
}{1
4
}%
\end{example}

\end{problem}

\pagebreak
%%%%%%%%%%%%%%%%%%%%%%%%%%%%%%%%%%%%%%%%%%%%%%%%%%%%%%%%%%%%%%%%%%%%%%%%%%%%%%%%%%%%%%%%%%%

\begin{problem}{โจทย์สุดง่าย(มั้ง)}{standard input}{standard output}{1 seconds}{256 megabytes}{100}

นักเรียนคนหนึ่งนั่งอยู่ในห้องเรียนวิชาคณิตศาสตร์ แต่เมื่อคืนเขาทำงานหนักเกินไปหน่อยเลยนอนดึกทำให้ง่วงนอนจนหลับในห้องเรียน แต่เนื่องจากเขานั่งอยู่ในแถวหน้าๆ ทำให้อาจารย์เห็นเข้า อาจารย์จึงให้ยืนขึ้นแล้วตอบคำถามที่เขียนอยู่บนกระดาน

ให้ $a$ เป็นจำนวนเต็มบวก โดย $f(n , a)$ เป็นพหุนามที่มีดีกรีสูงสุดเป็น $2^{n}$ โดยมีนิยามว่า 
\begin{center}

$f(0 , a) = x - a$ 

$f(n , a) = (f(n-1 , a))^2 - a^2 + a$  

\end{center}

ในตอนแรกอาจารย์ให้หาสัมประสิทธิ์ของ $x^p$ ใน $f(n , a)$ เมื่อ $0 \leq p \leq 2^n$ แต่อาจารย์คิดว่าคงหนักเกินไปเลยให้หาสัมประสิทธิ์ของ $x^2$ พอ

แต่เนื่องจากเขาได้บอกอาจารย์ว่าถามครั้งเดียวมันง่ายเกินไป อาจารย์เลยถามทั้งหมด $t$ ครั้ง โดยเปลี่ยนคำถามเป็น
ในคำถามที่ $i$ ให้ค่า $a_i$ และ $n_i$ มาให้หาสัมประสิทธิ์ของ $x^2$ ใน $f(n_i , a_i)$ 

หมายเหตุ : คำตอบอาจมีขนาดใหญ่ให้ตอบเป็นเศษที่เกิดจากการหารคำตอบด้วย $10^9 + 7$
\InputFile
บรรทัดที่ $1$ รับจำนวนเต็ม $t$ แสดงถึงจำนวนคำถาม $(1 \leq t \leq 10^5)$

บรรทัดที่ $2$ ถึง $t + 1$ รับจำนวนเต็ม $n_i$ และ $a_i$ $(0 \leq n_i \leq10^{18} , 2 \leq a_i \leq 10^8)$
\OutputFile
มีจำนวน $t$ บรรทัด ซึ่งบรรทัดที่ $i$ แสดงคำตอบของคำถามที่ $i$ 
\Scoring
ชุดทดสอบจะถูกแบ่งเป็น 2 ชุด จะได้คะแนนในแต่ละชุดก็ต่อเมื่อโปรแกรมให้ผลลัพธ์ถูกต้องในชุดทดสอบย่อยทั้งหมด

\begin{description}

\item[ชุดที่ 1 (25 คะแนน)]  $1 \leq t \leq 10^2 , 0 \leq n_i \leq 10^3 , 2 \leq a_i \leq 10^3$ 

\item[ชุดที่ 2 (75 คะแนน)] ไม่มีเงื่อนไขเพิ่มเติม

\end{description}

\Examples

\begin{example}
\exmp{1
1 2
}{1
}%
\exmp{3
0 5
1 6
2 7
}{0
1
210
}%
\end{example}

\Note
\begin{note}
\textbf{ตัวอย่างที่ 1}

$f(1, 2) = x^2 -4x+ 2$

\textbf{ตัวอย่างที่ 2}

$f(0,5)=x - 5$

$f(1,6)=x^2 - 12 x + 6$

$f(2,7)= x^4 - 28 x^3 + 210 x^2 - 196 x + 7$
\end{note}

\end{problem}

\pagebreak
%%%%%%%%%%%%%%%%%%%%%%%%%%%%%%%%%%%%%%%%%%%%%%%%%%%%%%%%%%%%%%%%%%%%%%%%%%%%%%%%%%%%%%%%%%%

\begin{problem}{เนียนให้ผ่าน}{standard input}{standard output}{1 seconds}{256 megabytes}{100}


ไม่นะ อยู่ในช่วงที่สถานีรถไฟทูยาร์ปต้องได้รับการตรวจปริมาณผู้โดยสารขั้นต่ำประจำปี โดยทางสถานีรถไฟได้เก็บบันทึกปริมาณผู้โดยสารที่จะเอาไปให้สำนักงานใหญ่ดูหมดแล้ว แต่ปรากฏว่าบันทึกบางส่วนนั้นมีจำนวนผู้โดยสารไม่ผ่านเกณฑ์ที่จะต้องมากกว่า 50\% ของความจุผู้โดยสารสูงสุด ถ้าสำนักงานใหญ่เห็นสถานีจะต้องโดนปิดอย่างแน่นอน ดังนั้นทางสถานีรถไฟจึงคิดแผนการชั่วร้ายออกมาคือการรวมผลบันทึกปริมาณผู้โดยสารนั่นเอง!

	เนื่องจากจำนวนผู้โดยสารในบางบันทึกนั้นเกิน 50\% ไปมากพอที่จะเป็นตัวช่วยของบันทึกอื่นๆที่มีจำนวนผู้โดยสารไม่เพียงพอได้ แต่มีข้อแม้อยู่เล็กน้อย คือการรวมบันทึกนั้นจะต้องรวมบันทึกลำดับที่อยู่ติดกันเท่านั้น เพราะว่าวันที่ของไฟล์ที่รวมแล้วจะได้ไม่โดดไปมาซึ่งจะทำให้สำนักงานใหญ่สงสัย และต้องพยายามให้จำนวนครั้งที่รวมน้อยที่สุดด้วย เพราะจะได้มีบันทึกหลายๆบันทึกไปส่งให้สำนักงานใหญ่ดูได้

	จงหาจำนวนบันทึกที่มากที่สุดหลังจากรวมแล้ว ทางสถานีรถไฟทูยาร์ปจะได้ทราบว่าตนเองมีโอกาสรอดเนียนจากสำนักงานใหญ่เท่าไหร่ 

\InputFile
บรรทัดแรก ระบุจำนวนเต็ม $ n,m$ $( 1\leq n\leq10^5 ; 1\leq m\leq10^6)$ โดยที่ $n$ แทนจำนวนบันทึกและ $m$ แสดงความจุผู้โดยสารสูงสุดโดย $m$ จะเป็นเลขคู่เสมอ

บรรทัดที่สอง ระบุจำนวนเต็ม $n$ ตัว ระบุ $X_1, X_2, ..., X_n$ ($1\leq X_i\leq m
)$ โดยที่  $X_i$ แสดงจำนวนผู้โดยสารในบันทึกลำดับที่ $i$ รับประกันว่าหากรวมทุกบันทึกเข้าด้วยกันแล้วจะมีจำนวนผู้โดยสารเกิน 50\% 

\OutputFile
มีทั้งหมด $1$ บรรทัดระบุจำนวนบันทึกที่มากที่สุดหลังจากรวมแล้วโดยที่ทุกบันทึกจะต้องมีจำนวนผู้โดยสารมากกว่า 50\%

\Scoring
ชุดทดสอบจะถูกแบ่งเป็น 2 ชุด จะได้คะแนนในแต่ละชุดก็ต่อเมื่อโปรแกรมให้ผลลัพธ์ถูกต้องในชุดทดสอบย่อยทั้งหมด

\begin{description}

\item[ชุดที่ 1 (30 คะแนน)] จะมี $ 1\leq n\leq10^3$

\item[ชุดที่ 2 (70 คะแนน)] ไม่มีเงื่อนไขเพิ่มเติม

\end{description}

\Examples

\begin{example}
\exmp{5 100
60 48 20 90 49
}{2}%
\exmp{12 100
60 48 40 56 59 57 45 48 51 52 53 54
}{6}%
\end{example}

\Note
\begin{note}
\textbf{ตัวอย่างที่ 1}

รวมบันทึกที่ $1,2$ ได้ $108/200$ คน

รวมบันทึกที่ $3,4,5$ ได้ $159/300$ คน

\textbf{ตัวอย่างที่ 2}

รวมบันทึกที่ $1,2$ ได้ $108/200$ คน

รวมบันทึกที่ $3,4,5$ ได้ $115/300$ คน

รวมบันทึกที่ $6,7$ ได้ $102/200$ คน

รวมบันทึกที่ $8,9,10$ ได้ $151/300$ คน

บันทึกที่ $11,12$ ไม่ต้องรวม

\end{note}


\end{problem}

\pagebreak
%%%%%%%%%%%%%%%%%%%%%%%%%%%%%%%%%%%%%%%%%%%%%%%%%%%%%%%%%%%%%%%%%%%%%%%%%%%%%%%%%%%%%%%%%%%

\begin{problem}{ab gift}{standard input}{standard output}{1 second}{256 megabytes}{100}

วินนี่ร้อนรนมากๆ เนื่องจากต้องการของขวัญให้หวานใจ โดยใช้เงินตัวเองซื้อแต่เงินไม่ใช่ปัญหา วินนี่ต้องการให้ได้มูลค่ามากที่สุดต่างหาก!

วินนี่ต้องการจะซื้อของขวัญโดยมีของขวัญทั้งหมด $N$ ชิ้น แต่ละชิ้นที่ $i$ จะมีมูลค่า $a_i$ และ $b_i$
โดยต้องการซื้อของขวัญให้ได้มากที่สุด แต่ขี้เกียจเลือกมากเพราะของเยอะ จึงจะซื้อเป็นช่วงของของขวัญจาก $l$ ถึง $r$ $( 1 \leq l \leq r \leq N )$ 
\begin{itemize}

\item โดยมูลค่าจากของที่ได้ทั้งหมดคือ  $a_l \cdot b_{l+1} \cdot b_{l+2} \cdot ... \cdot b_r +
a_{l+1} \cdot b_l \cdot b_{l+2} \cdot ... \cdot b_r +...+ a_r \cdot b_l \cdot b_{l+1} \cdot b_{l+2} \cdot ... \cdot b_{r-1}$ 
\item นอกจากนี้แล้ว วินนี่ได้สังเกตุว่า ทุกๆ ของขวัญชิ้นที่ $i$ จะมี $a_i+10000b_i = 10000$ เสมอ
\end{itemize}

ด้วยความร้อนรน วินนี่จึงไม่สามารถคิดได้อย่างที่เป็น จึงกลัวว่าจะไม่สามารถ เลือกได้ดีที่สุด จึงวานคุณให้มาช่วยคิดมูลค่ามากที่สุดให้เขาหน่อย! 

\InputFile
ข้อมูลนำเข้ามีทั้งหมด $N+2$ บรรทัด

บรรทัดแรก มีจำนวนเต็ม $N$ $( 1 \leq N \leq 10^6 )$ 

อีก $N$ บรรทัดต่อมา มี จำนวนเต็ม 2 จำนวน ระบุ $a_i$ และ $10000b_i$
$( 1 \leq a_i \leq 9999 )$ และ $( 1 \leq 10000 * b_i \leq 9999 ) $

\OutputFile
แสดงผลเป็นจำนวนเต็ม 1 จำนวน แทน 10000 เท่าของ มูลค่าสูงที่สุดที่ เป็นไปได้โดยปัดเศษลง

\Scoring
ชุดทดสอบจะถูกแบ่งเป็น 2 ชุด จะได้คะแนนในแต่ละชุดก็ต่อเมื่อโปรแกรมให้ผลลัพธ์ถูกต้องในชุดทดสอบย่อยทั้งหมด

\begin{description}

\item[ชุดที่ 1  (20 คะแนน)] จะมี $1 \leq N \leq 1000$

\item[ชุดที่ 2 (80 คะแนน)] ไม่มีเงื่อนไขเพิ่มเติมจากโจทย์

\end{description}

\Example

\begin{example}
\exmp{3
3000 7000
4000 6000
3500 6500}{47000000}%
\end{example}

\Note

เลือกของชิ้นที่ 2 และ 3 จะได้มากที่สุด

\end{problem}

\end{document}
