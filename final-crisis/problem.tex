\documentclass[11pt,a4paper]{article}

\newcommand{\tumsoTime}{09:00 น. - 12:00 น.}
\newcommand{\tumsoRound}{1}

\usepackage{../tumso}

\begin{document}

\begin{problem}{Final crisis}{standard input}{standard output}{0.5 seconds}{256 megabytes}{100}

ใกล้สอบปลายภาคแล้ว! ถึงเวลาที่เทพเอิร์ธจะต้องเริ่มอ่านหนังสือสอบ ด้วยความสามารถของเทพเอิร์ธ เขาอ่านหนังสือจบอย่างรวดเร็ว เหลืออยู่แค่สองวิชาที่ท่านเทพเอิร์ธไม่ชอบ คือ วิชาชีววิทยา กับวิชาประวัติศาสตร์

 หนังสือวิชาชีววิทยามีทั้งหมด $n$ เล่มและหนังสือวิชาประวัติศาสตร์มีทั้งหมด $m$ เล่ม หนังสือวิชาชีวะหนา $X_1, X_2, ..., X_n $ หน้า และหนังสือวิชาประวัติศาสตร์หนา $Y_1, Y_2, ..., Y_m$ หน้า เทพเอิร์ธเป็นคนที่มีสมาธิสูงมาก เมื่อเริ่มอ่านหนังสือเล่มไหนแล้วเขาต้องอ่านจนจบเล่ม และท่านเทพเอิร์ธต้องอ่านหนังสือเรียงจากเล่มที่ $1$ ไปเล่มที่ $n$ เพราะถ้าไม่อ่านเล่มที่ $1$ ก่อน ก็จะอ่านเล่มที่ $2$ ไม่รู้เรื่อง แต่เทพเอิร์ธตั้งใจเรียนในห้องทำให้เขาข้ามไปเริ่มอ่านวิชาชีวะที่เล่ม $a$ และวิชาประวัติศาสตร์ที่เล่ม $b$ ได้เลย และคุณครูก็ได้บอกว่าวิชาชีวะจะสอบถึงแค่เล่มที่ $c$ ส่วนวิชาประวัติศาสตร์จะสอบถึงเล่มที่ $d$

เนื่องจากท่านเทพเอิร์ธเกลียดทั้งสองวิชาพอๆกัน เขาจึงตั้งกฎกับตัวเองว่าเมื่ออ่านหนังสือจบเล่มนึงแล้ววิชาที่เขาจะอ่านต่อคือวิชาที่อ่านแล้วจำนวนหน้าสะสมจะน้อยกว่า เช่น อ่านชีวะมา $10$ หน้าแล้ว เล่มต่อไปมี $5$ หน้า อ่านประวัติศาสตร์มา $12$ หน้าแล้ว เล่มต่อไปมี $2$ หน้า เทพเอิร์ธจะเลือกอ่านประวัติศาสตร์ก่อนเพราะ $12+2 < 10+5$ ถ้าเท่ากันจะเลือกอ่านวิชาไหนก่อนก็ได้

เทพเอิร์ธเป็นคนขี้เหงา เทพเอิร์ธจึงตั้งโจทย์ให้คุณมานั่งทำเป็นเพื่อนเขา เทพเอิร์ธถามคุณว่า ถ้าเขาอ่านหนังสือทั้งสองวิชารวมกัน $k$ เล่มแล้ว จำนวนหน้าสะสมของวิชาที่อ่านไปมากกว่าคือเท่าไหร่ เทพเอิร์ธคิดว่าคงต้องอ่านหนังสือไปอีกนาน เขาจึงตัดสินใจถามคุณ $q$ ครั้งคุณจะได้นั่งเป็นเพื่อนเขานานๆ แต่คุณขี้เกียจนั่งตอบคำถามทั้งหมด คุณจึงตัดสินใจจะตอบคำถามทั้งหมดภายใน $0.5$ วินาทีแล้วรีบหนีไปเล่นเกม


\InputFile
บรรทัดแรก ระบุจำนวนเต็ม $ n,m,q$ $( 1\leq n,m,q\leq10^5)$

บรรทัดที่สอง ระบุจำนวนเต็ม $n$ ตัว ระบุ $X_1, X_2, ..., X_n$ ($1\leq X_i\leq20000
)$

บรรทัดที่สาม ระบุจำนวนเต็ม $m$ ตัว ระบุ $Y_1,Y_2,...,Y_m$ $(1\leq Y_i\leq 20000)$

อีก $q$ บรรทัด ระบุจำนวนเต็ม $a$ $b$ $c$ $d$ $k$ $(1\leq a\leq c\leq n $ ; $1\leq b\leq d\leq n $ ; $ 1\leq k\leq c-a+d-b+2)$


\OutputFile
มีทั้งหมด $q$ บรรทัดระบุคำตอบของแต่ละคำถาม

\Scoring
ชุดทดสอบจะถูกแบ่งเป็น 2 ชุด จะได้คะแนนในแต่ละชุดก็ต่อเมื่อโปรแกรมให้ผลลัพธ์ถูกต้องในชุดทดสอบย่อยทั้งหมด

\begin{description}

\item[ชุดที่ 1 (30 คะแนน)] จะมี $ 1\leq n,m,q\leq10^3$

\item[ชุดที่ 2 (70 คะแนน)] ไม่มีเงื่อนไขเพิ่มเติม

\end{description}

\Examples

\begin{example}
\exmp{5 6 2
1 5 3 7 2
4 7 2 5 9 2
1 1 3 3 4
1 3 4 5 7
}{9
16}%
\end{example}

\Note
\begin{note}
\textbf{คำถามที่ 1}

จำนวนหน้าวิชาชีวะคือ 1 5 3

จำนวนหน้าวิชาประวัติศาสตร์คือ 4 7 2

เลือกอ่าน ชีวะ(1) ประวัติ(4) ชีวะ(6) ชีวะ(9) ประวัติ(11) ประวัติ(13) ตามลำดับ

เมื่ออ่านไป 4 เล่มจะอ่านชีวะไป 9 หน้า อ่านประวัติไป 4 หน้า จึงตอบ 9

\textbf{คำถามที่ 2}

จำนวนหน้าวิชาชีวะคือ 1 5 3 7

จำนวนหน้าวิชาประวัติศาสตร์คือ 2 5 9

เลือกอ่าน ชีวะ(1) ประวัติ(2) ชีวะ(6) ประวัติ(7) ชีวะ(9) ประวัติ(16) ชีวะ(16) ตามลำดับ

เมื่ออ่านไป 7 เล่ม จะอ่านชีวะไป 16 หน้า อ่านประวัติไป 16 หน้า จึงตอบ 16
\end{note}


\end{problem}

\end{document}
