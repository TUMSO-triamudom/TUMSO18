\documentclass[11pt,a4paper]{article}

\newcommand{\tumsoTime}{09:00 น. - 12:00 น.}
\newcommand{\tumsoRound}{1}

\usepackage{../tumso}

\begin{document}

\begin{problem}{ab gift }{standard input}{standard output}{1 second}{256 megabytes}{100}

วินนี่ร้อนรนมากๆ เนื่องจากต้องการของขวัญให้หวานใจ โดย ใช้เงิน ตัวเองซื้อ แต่เงินไม่ใช่ปัญหา วินนี่ต้องการให้ได้มูลค่ามากที่สุดต่างหาก!

วินนี่ต้องการจะซื้อของขวัญ โดยมี ของขวัญทั้งหมด $N$ ชิ้น แต่ละชิ้นที่ $i$ จะมีมูลค่า $a_i$ และ  $b_i$
โดยต้องการซื้อ ของขวัญ ให้ได้มากที่สุด แต่ ขี้เกียจเลือกมากเพราะของเยอะ จึงจะซื้อเป็น ช่วงของของขวัญจาก $l$ ถึง $r$ $( 1 \leq l \leq r \leq N )$ 
\begin{itemize}

\item โดยมูลค่าจากของที่ได้ทั้งหมดคือ  $a_l \cdot b_{l+1} \cdot b_{l+2} \cdot ... \cdot b_r +
a_{l+1} \cdot b_l \cdot _b_{l+2} \cdot ... \cdot b_r +...+ a_r \cdot b_l \cdot b_{l+1} \cdot b_{l+2} \cdot ... \cdot b_{r-1}$ 
\item นอกจากนี้แล้ว วินนี่ได้สังเกตุว่า ทุกๆ ของขวัญชิ้นที่ $a_i+10000b_i = 10000$ เสมอ
\end{itemize}

ด้วยความร้อนรน วินนี่จึงไม่สามารถคิดได้อย่างที่เป็น จึงกลัวว่าจะไม่สามารถ เลือกได้ดีที่สุด จึงวานคุณให้มาช่วยคิดมูลค่ามากที่สุดให้เขาหน่อย! 

\InputFile
ข้อมูลนำเข้ามีทั้งหมด $N+2$ บรรทัด

บรรทัดแรก มีจำนวนเต็ม $N$ $( 1 \leq N \leq 10^6 )$ 

อีก $N$ บรรทัดต่อมา มี จำนวนเต็ม 2 จำนวน ระบุ $a_i$ และ $10000b_i$
$( 1 \leq a_i \leq 9999 )$ และ $( 1 \leq 10000 * b_i \leq 9999 ) $

\OutputFile
แสดงผลเป็นจำนวนเต็ม 1 จำนวน แทน 10000 เท่าของ มูลค่าสูงที่สุดที่ เป็นไปได้โดยปัดเศษลง

\Scoring
ชุดทดสอบจะถูกแบ่งเป็น 2 ชุด จะได้คะแนนในแต่ละชุดก็ต่อเมื่อโปรแกรมให้ผลลัพธ์ถูกต้องในชุดทดสอบย่อยทั้งหมด

\begin{itemize}

\item \textbf{ชุดที่ 1  (20 คะแนน)} จะมี $1 \leq N \leq 1000$

\item \textbf{ชุดที่ 2 (80 คะแนน)} ไม่มีเงื่อนไขเพิ่มเติมจากโจทย์

\end{itemize}

\Example

\begin{example}
\exmp{3
3000 7000
4000 6000
3500 6500}{47000000}%
\end{example}

\Note

เลือกของชิ้นที่ 2 และ 3 จะได้มากที่สุด

\end{problem}

\end{document}
