\documentclass[11pt,a4paper]{article}

\newcommand{\tumsoTime}{09:00 น. - 12:00 น.}
\newcommand{\tumsoRound}{1}

\usepackage{../style/tumso}

\begin{document}

\begin{problem}{Math Math}{standard input}{standard output}{1 seconds}{256 megabytes}{100}

หลังจากคุณได้ช่วยวินนี่ผู้ร้อนรน ซื้อของมูลค่าสูงสุดได้สำเร็จ(มั้ยหว่า) วินนี่ก็แบกกล่องจำนวนมหาศาลไปให้หวานใจถึงหน้าบ้านเลย!

ทั้งปริมาณเงินที่เสียไป ทั้งความเหนื่อยที่แบกของมา ทั้งหมดต้องคุ้มค่าอย่างมาก เมื่อหวานใจของวินนี่ได้ เห็นสิ่งที่วินนี่ทำลงไปเพื่อเธอ
เธอยิ้มจนแก้มปริ กระโดดกอดวินนี่และหอมแก้มวินนี่เป็นจำนวนเท่ากับ k ครั้ง แต่การหอมแก้ม 1 ครั้งนั้นย่อมทำให้วินนี่มีความสุขมากกว่า 1 หน่วยเป็นแน่แท้ 
โดยหากหวานใจวินนี่หอมแก้มวินนี่ เป็นจำนวน k ครั้ง วินนี่จะมีปริมาณความสุขเท่ากับผลรวมเลขโดด $\frac{10^{126k + 3} + 143}{127}$ หน่วยเลยทีเดียว

หลังจากเหตุการณ์ผ่านไป วินนี่มีความสุขมากก แต่ก็อยากรู้ว่ามีความสุขปริมาณเท่าไหร่ จึงได้บอก ปริมาณ k เพื่อให้คุณมาหาปริมาณความสุขรวมให้วินนี่หน่อย!

\InputFile
ข้อมูลนำเข้ามีทั้งหมด $T$ บรรทัด แสดงถึงจำนวนพหุจักรวาลที่ เหตุการณ์นี้เกิดขึ้น $(1 \leq T \leq 10^5)$

บรรทัดถัดมาอีก $T$ บรรทัดประกอบด้วย $k_i$ แทนจำนวนครั้งการหอมแก้มในแต่ละจักรวาล $(0 \leq k_i \leq 10^{15})$ 

\OutputFile
มีทั้งหมด $T$ บรรทัด แต่ละบรรทัดประกอบปริมาณความสุขใน จักรวาลนั้นๆ

\Scoring
ชุดทดสอบจะถูกแบ่งเป็น 3 ชุด จะได้คะแนนในแต่ละชุดก็ต่อเมื่อโปรแกรมให้ผลลัพธ์ถูกต้องในชุดทดสอบย่อยทั้งหมด

\begin{description}

\item[ชุดที่ 1 (15 คะแนน)] จะมี $k_i = 0$

\item[ชุดที่ 2 (35 คะแนน)] จะมี $\max_{i=1}^T(k_i) \cdot T \leq 10^6$

\item[ชุดที 3 (50 คะแนน)] ไม่มีเงื่อนไขเพิ่มเติม

\end{description}

\Examples

\begin{example}
\exmp{1
1}{576
}%
\end{example}

\pagebreak

\Note

สำหรับ $k = 1$

$\frac{10^{126\cdot1 + 3} + 143}{127} = 787401574803149606299212598425196850393700787401574803149606299212598425\\1968503937007874015748031496062992125984251968503937009$

ซึ่ง 

$7 + 8 + 7 + 4 + 0 + 1 + 5 + 7 + 4 + 8 + 0 + 3 + 1 + 4 + 9 + 6 + 0 + 6 + 2 + 9 + 9 + 2 + 1 + 2 + 5 + 9 + 8 + \\4 + 2 + 5 + 1 + 9 + 6 + 8 + 5 + 0 + 3 + 9 + 3 + 7 + 0 + 0 + 7 + 8 + 7 + 4 + 0 + 1 + 5 + 7 + 4 + 8 + 0 + 3 + \\1 + 4 + 9 + 6 + 0 + 6 + 2 + 9 + 9 + 2 + 1 + 2 + 5 + 9 + 8 + 4 + 2 + 5 + 1 + 9 + 6 + 8 + 5 + 0 + 3 + 9 + 3 + \\7 + 0 + 0 + 7 + 8 + 7 + 4 + 0 + 1 + 5 + 7 + 4 + 8 + 0 + 3 + 1 + 4 + 9 + 6 + 0 + 6 + 2 + 9 + 9 + 2 + 1 + 2 + \\5 + 9 + 8 + 4 + 2 + 5 + 1 + 9 + 6 + 8 + 5 + 0 + 3 + 9 + 3 + 7 + 0 + 0 + 9 = 576$

\end{problem}

\end{document}
