\documentclass[11pt,a4paper]{article}

\newcommand{\tumsoTime}{09:00 น. - 12:00 น.}
\newcommand{\tumsoRound}{1}

\usepackage{../tumso}

\begin{document}

\begin{problem}{โจทย์สุดง่าย(มั้ง)}{standard input}{standard output}{1 seconds}{64 megabytes}{100}

นักเรียนคนหนึ่งนั่งอยู่ในห้องเรียนวิชาคณิตศาสตร์ แต่เมื่อคืนเขาทำงานหนักเกินไปหน่อยเลยนอนดึกทำให้ง่วงนอนจนหลับในห้องเรียน แต่เนื่องจากเขานั่งอยู่ในแถวหน้าๆ ทำให้อาจารย์เห็นเข้า อาจารย์จึงให้ยืนขึ้นแล้วตอบคำถามที่เขียนอยู่บนกระดาน

ให้ $a$ เป็นจำนวนเต็มบวก โดย $f(n , a)$ เป็นพหุนามที่มีดีกรีสูงสุดเป็น $2^{n}$ โดยมีนิยามว่า 
\begin{center}

$f(0 , a) = x - a$ 

$f(n , a) = (f(n-1 , a))^2 - a^2 + a$  

\end{center}

ในตอนแรกอาจารย์ให้หาสัมประสิทธิ์ของ $x^p$ ใน $f(n , a)$ เมื่อ $0 \leq p \leq 2^n$ แต่อาจารย์คิดว่าคงหนักเกินไปเลยให้หาสัมประสิทธิ์ของ $x^2$ พอ

แต่เนื่องจากเขาได้บอกอาจารย์ว่าถามครั้งเดียวมันง่ายเกินไป อาจารย์เลยถามทั้งหมด $t$ ครั้ง โดยเปลี่ยนคำถามเป็น
ในคำถามที่ $i$ ให้ค่า $a_i$ และ $n_i$ มาให้หาสัมประสิทธิ์ของ $x^2$ ใน $f(n_i , a_i)$ 

หมายเหตุ : คำตอบอาจมีขนาดใหญ่ให้ตอบเป็นเศษที่เกิดจากการหารคำตอบด้วย $10^9 + 7$
\InputFile
บรรทัดที่ $1$ รับจำนวนเต็ม $t$ แสดงถึงจำนวนคำถาม $(1 \leq t \leq 10^5)$

บรรทัดที่ $2$ ถึง $t + 1$ รับจำนวนเต็ม $n_i$ และ $a_i$ $(0 \leq n_i \leq10^{18} , 2 \leq a_i \leq 10^8)$
\OutputFile
มีจำนวน $t$ บรรทัด ซึ่งบรรทัดที่ $i$ แสดงคำตอบของคำถามที่ $i$ 
\Scoring
ชุดทดสอบจะถูกแบ่งเป็น 2 ชุด จะได้คะแนนในแต่ละชุดก็ต่อเมื่อโปรแกรมให้ผลลัพธ์ถูกต้องในชุดทดสอบย่อยทั้งหมด

\begin{description}

\item[ชุดที่ 1 (25 คะแนน)]  $1 \leq t \leq 10^2 , 0 \leq n_i \leq 10^3 , 2 \leq a_i \leq 10^3$ 

\item[ชุดที่ 2 (75 คะแนน)] ไม่มีเงื่อนไขเพิ่มเติม

\end{description}

\Examples

\begin{example}
\exmp{1
1 2
}{1
}%
\exmp{3
0 5
1 6
2 7
}{0
1
210
}%
\end{example}

\Note
\begin{note}
\textbf{ตัวอย่างที่ 1}

$f(1, 2) = x^2 -4x+ 2$

\textbf{ตัวอย่างที่ 2}

$f(0,5)=x - 5$

$f(1,6)=x^2 - 12 x + 6$

$f(2,7)= x^4 - 28 x^3 + 210 x^2 - 196 x + 7$
\end{note}

\end{problem}

\end{document}
