\documentclass[11pt,a4paper]{article}

\newcommand{\tumsoTime}{09:00 น. - 12:00 น.}
\newcommand{\tumsoRound}{1}

\usepackage{../style/tumso}

\begin{document}

\begin{problem}{TU Lympics}{standard input}{standard output}{1 seconds}{64 megabytes}{100}

ในที่สุดก็ถึงเวลาสำหรับงานกีฬาสี ทุกๆ คนต่างมีภาระหน้าที่ที่ต้องรับผิดชอบ ไม่ว่าจะเป็นทั้งพาเหรด สแตนด์ และอื่นๆ ทั้งเบื้องหน้าและเบื้องหลัง นอกจากนี้ โรงเรียนก็ได้คิดกฎและกติกาการแข่งขันแบบใหม่ ในหลายๆ ชนิดกีฬา รวมไปถึงการแข่งขันวิ่งด้วย โดยรายละเอียดของกฎและกติกาแบบใหม่นั้น มีดังนี้

ในการแข่งขันนั้นจะมีผู้เข้าแข่งขันทั้งหมด $N$ คน แต่ละคนจะมีหมายเลขประจำตัวตั้งแต่ $1$ ถึง $N$ โดยจะแข่งขันกันทั้งหมด $M$ รอบ ผู้ชนะคือผู้ที่ได้รับคะแนนสูงที่สุด หากมีผู้ที่ได้คะแนนสูงสุดมากกว่า $1$ คน ผู้ชนะคือผู้ที่มีหมายเลขประจำตัวน้อยที่สุด โดยวิธีการคิดคะแนน จะคิดเป็นรอบต่อรอบ กล่าวคือ ในแต่ละรอบ เราจะนำเวลาที่ผู้เข้าแข่งแต่ละคนใช้ในรอบนั้นๆ มาเรียงลำดับจากน้อยไปมาก ผู้เข้าแข่งขันคนใดที่ใช้เวลาน้อยที่สุด ก็จะได้คะแนนมากที่สุด คนถัดมาก็จะได้คะแนนลดลง เรื่อยไปจนถึงอันดับสุดท้าย ในกรณีที่มีผู้เข้าแข่งขันมากกว่าหนึ่งคนใช้เวลาเท่ากัน ผู้เข้าแข่งขันคนใดที่มีหมายเลขประจำตัวที่น้อยกว่า จะได้อันดับที่ดีกว่า

ในการแข่งขันครั้งนี้นั้น เพื่อนของคุณเป็นหนึ่งในผู้เข้าแข่งขัน และเขานั้นก็อยากจะรู้ว่าในการแข่งขันครั้งนี้นั้น เขาได้อันดับที่เท่าไหร่ แต่ด้วยเหตุการณ์ทางเทคนิคบางประการ จึงทำให้ผลการแข่งขันนั้นเกิดความล่าช้า เพื่อนของคุณทราบว่าคุณนั้นมีความสามารถในการเขียนโปรแกรมระดับยอดเยี่ยม จึงได้มาขอให้คุณไปช่วยฝ่ายสถิติ เขียนโปรแกรมเพื่อจัดอันดับในการแข่งขันวิ่งครั้งนี้ เพื่อที่เขาจะได้รู้ว่าเขานั้นได้อันดับที่เท่าไหร่ และในบางครั้ง เขานั้นก็อยากจะรู้ด้วยว่า เขานั้นได้คะแนนเท่าไหร่ 

\InputFile
ข้อมูลนำเข้านั้นมีด้วยกัน $N+2$ บรรทัด 

บรรทัดที่ $1$ ประกอบด้วยจำนวนเต็มบวก $N$ $M$ $X$ และ $T$ แทนจำนวนผู้เข้าแข่งขัน จำนวนรอบในการแข่งขัน หมายเลขประจำตัวผู้เข้าแข่งขันของเพื่อนของคุณ และประเภทของคำถามตามลำดับ
	
บรรทัดที่ $2$ ประกอบด้วยจำนวนเต็มบวกทั้งหมด $N$ ตัว แทนคะแนนที่ผู้เข้าแข่งขันจะได้รับในแต่ละรอบ เมื่อได้ลำดับที่ $1$ ไปจนถึงลำดับที่ $N$ ตามลำดับ โดยรับประกันว่าคะแนนจะเรียงลำดับจากมากไปน้อยเสมอ และจะไม่ซ้ำกัน

บรรทัดที่ $2+i$ ประกอบด้วยจำนวนเต็มบวกทั้งหมด $M$ ตัวแทนเวลาที่ผู้เข้าแข่งขันคนที่ $i$ ใช้ในรอบที่ $1$ จนถึงรอบที่ $M$ ในหน่วยวินาที  $(1 \leq i \leq N )$ 

และรับประกันว่าเวลาที่ผู้เข้าแข่งขันใช้ในแต่ละรอบจะไม่เกิน $10^9$ วินาที คะแนนที่ผู้เข้าแข่งขันจะได้ในแต่ละรอบจะมีค่าไม่เกิน $2 \cdot 10^8$ คะแนน

\OutputFile
มี $1$ บรรทัด 

หาก $T = 1$ ให้แสดงจำนวนเต็มบวก $1$ ตัว คืออันดับที่เพื่อนของคุณได้ 

หาก $T = 2$ ให้แสดงจำนวนเต็มบวก $2$ ตัว คืออันดับที่เพื่อนของคุณได้ และคะแนนที่เขาได้รับตามลำดับ โดยเว้นช่องว่าง $1$ ช่องระหว่างจำนวนเต็มบวกทั้งสอง

\Scoring
ชุดทดสอบจะถูกแบ่งเป็น $2$ ชุด จะได้คะแนนในแต่ละชุดก็ต่อเมื่อโปรแกรมให้ผลลัพธ์ถูกต้องในชุดทดสอบย่อยทั้งหมด

\begin{description}

\item[ชุดที่ 1 (30 คะแนน)]  $N \leq 200$ , $M \leq 5$ , $1 \leq X \leq N$ , $T=1$

\item[ชุดที่ 2 (70 คะแนน)]  $N\leq20000$  , $M\leq5$ , $1 \leq X \leq N $ , $T=1$ หรือ $2$

\end{description}

\Examples

\begin{example}
\exmp{5 3 4 1
10 7 5 2 1
12 24 18
17 20 19
30 12 13
10 15 22
20 22 21
}{2
}%
\exmp{7 4 2 2
19 18 16 12 8 5 2
20 11 21 32
21 20 20 17
20 14 19 20
17 30 19 22
25 18 15 26
40 10 20 30
25 22 22 18
}{2 51
}%
\end{example}

\end{problem}

\end{document}
