\documentclass[11pt,a4paper]{article}

\newcommand{\tumsoTime}{13:00 น. - 16:00 น.}
\newcommand{\tumsoRound}{2}

\usepackage{../style/tumso}

\begin{document}

\begin{problem}{Forest Resorts}{standard input}{standard output}{1 second}{256 megabytes}{100}

คุณเป็นเจ้าของที่ที่ใหญ่มาอยู่ที่หนึ่ง เต็มไปด้วยต้นไม้มากมาย พื้นที่นี้มีที่ที่สามารถสร้างรีสอร์ทได้ทั้งหมด $N (1 \leq N \leq 10^5)$ จุด แต่ยังไม่ได้สร้างขึ้นเพราะต้องถอนต้นไม้ออกเสียก่อน นอกจากนี้ คุณมีแผนที่จะวางทางเชื่อมบนจุดเหล่านี้ทั้งหมด $N - 1$ ทางเชื่อม โดยจะเชื่อมจุดเหล่านี้นเข้าหากัน โดยถ้าหากสร้างทางเชื่อมจนครบ จะสามารถเดินจากจุดหนึ่ง ไปอีกจุดได้เสมอโดยมีเพียงเส้นทางเดียวเท่านั้นที่เดินได้

คุณมีแผนที่จะสร้างรีสอร์ททั้งหมด $Q (1 \leq Q \leq 10^5)$ แผน แต่ละแผน คุณจะสามารถสร้างรีสอร์ทได้ $K (1 \leq K \leq N)$ หลัง ซึ่งเนื่องจากคุณต้องการให้รีสอร์ทที่คุณสร้าง สามารถเดินหากันได้ ทางเชื่อมที่จำเป็นในการเดินหากันของรีสอร์ทที่คุณเลือกจึงต้องถูกสร้างขึ้น โดยคุณจะพยายามสร้างทางเชื่อมให้น้อยที่สุดที่ยังตรงตามเงื่อนไข เพราะคุณก็ไม่ได้รวย

สำหรับทางเชื่อมที่ไม่ได้สร้างขึ้น จะต้องให้ญาติของคุณไป ถึงแม้ว่าทางเชื่อมจะไม่ได้ถูกสร้าง คุณก็ไม่อยากให้ญาติคุณไปฟรีๆ ดังนั้นในแต่ละแผนการสร้าง คุณจึงต้องการรู้ว่าสร้างทางเชื่อมได้มากสุดกี่ทาง หากคุณเลือกสร้างรีสอร์ทตรงไหนก็ได้ (อาจซ้ำที่กันก็ได้) และเนื่องจากเป็นเพียงแผน ให้เสมือนว่ายังไม่ได้มีการสร้างใดๆ เกิดขึ้นในทุกๆ แผนการสร้าง

\InputFile
บรรทัดแรก ระบุจำนวนเต็ม $(1 \leq N \leq 10^5)$ แทนจำนวนจุดที่สามารถสร้างรีสอร์ททั้งหมด

บรรทัดถัดไปอีก $N - 1$ บรรทัด ระบุจำนวนเต็ม $u, v (1 \leq u, v \leq 10^5, u \neq v)$ แทนว่ามีแผนจะวางทางเชื่อมระหว่างจุดสร้างรีสอร์ท $u$ และ $v$

บรรทัดที่ $N + 1$ ระบุจำนวนเต็ม $Q$ แทนจำนวนแผนการสร้างรีสอร์ท

บรรทัดถัดไปอีก $Q$ บรรทัด ระบุจำนวนเต็ม $K$ แทนจำนวนรีสอร์ทที่สร้างได้ในแผนครั้งนั้น

\OutputFile
มีทั้งหมด $Q$ บรรทัด ระบุคำตอบของแต่ละคำถาม

\Scoring
ชุดทดสอบจะถูกแบ่งเป็น 3 ชุด จะได้คะแนนในแต่ละชุดก็ต่อเมื่อโปรแกรมให้ผลลัพธ์ถูกต้องในชุดทดสอบย่อยทั้งหมด

\begin{itemize}
\item \textbf{ชุดที่ 1 (25 คะแนน)} จะมี $1 \leq N, Q\leq 7$

\item \textbf{ชุดที่ 2 (25 คะแนน)} จะมี $1 \leq N, Q\leq 10^3$

\item \textbf{ชุดที่ 3 (50 คะแนน)} ไม่มีเงื่อนไขเพิ่มเติม
\end{itemize}

\Example

\begin{example}
\exmp{6
1 2
1 3
2 4
2 5
3 6
2
2
3
}{4
5}%
\end{example}

\Note
\textbf{คำถามที่ 1}

สร้างรีสอร์ทที่ตำแหน่ง 4 และ 6 ทำให้ต้องสร้างทางเชื่อมที่ 1, 2, 3, 5 ตามลำดับที่ให้มา

\textbf{คำถามที่ 2}

สร้างรีสอร์ทที่ตำแหน่ง 4, 5 และ 6 ทำให้ต้องสร้างทางเชื่อมทั้งหมด

\end{problem}

\end{document}
